\documentclass[a4paper, 11pt, oneside]{book}
\usepackage{fontspec}
\usepackage{lmodern}
\usepackage[czech]{babel}
\usepackage{csquotes}
\usepackage{url}
\usepackage{todonotes}
\usepackage{xunicode}


\usepackage[
   backend=biber
  ,style=iso-authoryear
  ,sortlocale=cs_CZ
  ,autocite=footnote
  ,maxnames=2
  ,minnames=1
  ,urldate=long
  ,spacecolon=false
  ]
  {biblatex}


\addbibresource{bibliografie.bib}

\newcommand{\td}[2][]{
	{\todo[size=\footnotesize]{#2}}
}

\newcommand\ex{\textsf}

\begin{document}

	\part{Teoretické pojednání}

	\chapter{Wordnet}

		\section{Motivace vzniku}
			Od počátků snah o zpracování přirozeného jazyka (NLP, natural language processing) bylo nutné poskytnout programu data o lexiku ve zpracovávaném textu, ať už ona data byla jakákoliv. Kupříkladu pro překlad se mělo za to, že stačí ekvivalentní dvojice ve zdrojovém a cílovém jazyce, později se přidal kontext v případě statistického strojového překladu spolu s dalšími informacemi, jako je například slovní druh. Tradičně se lexikální materiál ukládá způsobem nikoliv diametrálně odlišným od papírových slovníků určených pro lidské uživatele. Ty obvykle obsahují abecenedně (či podle jiného indexu \td{cit?}) seřazené jednotlivé záznamy s potřebnými informacemi o slovech, z nichž pak program může čerpat při zpracování textu.

			Jak uvádí \textcite{pala2013vceska}, uspořádání lexikálního materiálu v takovéto formě je sice vhodné pro člověka, ale nikoliv pro strojové zpracování, a to z několika důvodů. Kromě toho, že vyhledávání v abecedním seznamu je relativně pomalé \td{nejaka citace?, dohledat neco, jak takovy slovniky byly ulozeny...}, struktura tradičního slovníku kvůli onomu abecednímu řazení inherentně vzdaluje slova, jež člověk chápe jako nějakým způsobem blízká. Tato blízkost může vyplývat ze vztahu volné synonymie, antonymie, podřazenosti, nadřazenosti, etc. Pokud si tedy například uživatel výkladového slovníku nepříliš obeznámený s daným jazykem vyhledá určité heslo, dozví se sice pravděpodobně jeho význam, ale nebude schopen své znalosti prohlubovat dále zjištěním, kupříkladu jaké je slovo odpovídá opačnému významu.

			Dalším všeobecným problémem při využití tradičních slovníků k počítačovému zpracování jazyka je fakt, že lexikografové předpokládají u uživatele slovníku značné encyklopedické znalosti. Zařazují tak do slovníku jen informace dle jejich názoru důležité pro rozlišení (differentia specifica) a zařazující do kontextu či přiřazující k určité nadřazené třídě objektů (genus proximum). Vyhledá-li si tedy člověk ve Slovníku spisovného jazyka českého\td{citace} heslo \ex{vlk}, zjistí následující:

			\ex{\textbf{vlk: } psovitá šelma šedě (n. šedožlutě) zbarvená, žijící v Evropě, Asii a v Sev. Americe}

			Definice a priori předpokládá, že uživatel je obeznámen s tím, co je \ex{šelma} a co je \ex{pes}. Pokud takovou znalostí neslyne (což je vcelku představitelné například u cizince), je nucen si tato slova ve slovníku najít a podívat se na jejich definice (pomiňme nyní netriviální úkol převést slovo \ex{psovitá} na základní tvar \ex{pes}). Pokud nerozumí definicím ani nadřazených slov, musí pokračovat v hierarchii dále a dále. 

			Z uvedeného případu plyne, že jakkoliv je možné správným vyhledáváním hyperonym\footnote{nadřazené slovo} dospět k tomu, že \ex{vlk} je konkrétní entita našeho vesmíru, živá bytost o čtyřech končetinách, savec nějakým způsobem přibuzný se psovi, má šedou srst etc., je takový proces dosti komplikovaný. Případ s cizincem se sice nemusí zdát zcela relevantní, protože se dá předpokladat, že daný člověk má, byť v jiném jazyce, stejné základní znalosti předpokládané lexikografy jako člověk český. Situace je však dramaticky jiná u počítače. Na rozdíl od člověka totiž počítač nemá žádné předchozí znalosti, tudíž musí projít celým procesem popsaným výše, aby byl schopen kupříkladu určit, že \ex{vlk} může umřít (ježto je živá bytost). Protože však tradiční slovníky typu SSJČ byly vytvářené pro papírové médium, neobsahují žádné propojení ve stylu \textit{toto je odkaz na hyperonymum}, a počítač tudíž jen těžko může zjišťovat, na které vlastně slovo se to má podívat, aby se dobral podstaty pojmu \ex{vlk}.

			\subsection{Strojově čitelné slovníky}

				V zájmu automatizace vyhledávání ve slovníku vznikaly tzv. strojově čitelné slovníky\footnote{machine readable dictionary}, což je pojem souhrnně označující lexikální databáze. Podle množství informací, které taková databáze obsahuje, pak lze tyto dělit na slovníky, taxonomie a ontologie. Je evidentní, že obyčejný slovník neobsahuje oproti tradičnímu papírovému slovníku navíc žádné metainformace, takže je počítač při jeho užívání v podstatě omezen na elektronický listovač \parencite{miller1990introduction}. 

				Míra, jakou se strojově čitelný slovník odliší od pouhé zdigitalizované formu papírového slovníku a přiblíží se k pokročilé lexikální databázi, lze vyjádřit v několika stupních. V případě, že slovník má jednotlivé koncepty\td{nejakej link, kde budou koncepty/senses vysvetleny} uspořádány v hierarchii dle nadřazenosti--podřazenosti, lze jej označit za taxonomii, tedy systém s hlubší strukturou než pouze abecedním řazením hesel. 

				Dalším stupněm už je skutečná lexikální databáze, která má jednotlivé koncepty propojeny rozličnými vztahy, počínaje onou základní hyperonymií a hyponymií a pokračuje kupříkladu vztahy meronymie\footnote{vztah \textit{je částí}, tedy např. \ex{dveře} je meronymem \ex{trolejbusu}} či antonymie\footnote{protikladu}. Kromě vztahů mezi koncepty bude taková lexikální databáze obsahovat zřejmě i další informace, tedy nějaké kategorie slov, jejich popis, etc. Databáze tak popsaných konceptů propojených sémantickými vztahy může být nazývána ontologií.\parencite{garshoi2004metadata}

			\subsection{Od slovníků k Wordnetu}

				Výše uvedená opozice papírového slovníku a ontologie ilustruje rozdíly tradičního slovníku a počítačově zpracovatelné lexikální databáze. Už ze samotného konceptu takové databáze je evidentní jeden klíčový rozdíl -- tradiční slovníky, jsouce řazené abecedně, od sebe oddalují některá hesla, jež by bylo vhodné mít pohromadě \parencite{pala2013vceska}. Příkladem budiž \ex{kostra} a její části, např. \ex{lebka}. V SSČ\footnote{Slovník spisovné češtiny} i SSJČ se u \ex{lebky} uvádí, že jde o \ex{kostru hlavy}. Lze tedy s jistou rezervou tvrdit, že heslo obsahuje své holonymum\footnote{vztah opačný k meronymii; tedy např. \ex{dům} je holonymem pro \ex{okno}, \ex{dveře}, \ex{práh} etc.}, opačně to však již nefunguje. Z celkem evidentních důvodů nejsou u hesla \ex{kostra} uvedeny všechny její části. Tento příklad příhodně ukazuje i jistou nesystémovost tradičních slovníků, která je pro počítačové zpracování fatální. % a pak ty slovni druhy, vole!

				Naznačeny tedy byly vlastnosti, jež by databáze významů měla oproti tradičnímu slovníku mít, aby byla použitelná pro počítačové zpracování přirozeného jazyka. Především jde o systémovost vztahů. Hypero/hyponymie je vztah oboustranný, tudíž by mělo být možné se stejnou cestou dostat od nadřazeného slova k podřazenému a naopak. Dále jsou podstatné pojmenované sémantické vztahy mezi slovy. Díky nim je totiž možno jednoznačně určit, které slovo (či slova) je v takové databázi konkrétnímu slovu nadřazené, které je jeho specifikací, označením jeho částí, etc. 

				S touto myšlenkou tedy vznikl Wordnet - lexikální síť provázaná sémantickými vztahy, která by dle poznatků psycholingvistiky odrážela uspořádání lexikálního materiálu v lidském mozku (o tom v dalších kapitolách\td{ref}). \parencite{pala2013vceska} Zde by bylo na místě poznamenat, že ačkoliv se tak z odstavců výše může čtenáři jevit a i všeobecně je to často tvrzeno, Wordnet není ontologií v pravém slova smyslu, protože něco něco.. \url{https://en.wikipedia.org/wiki/WordNet#WordNet_as_a_lexical_ontology} \td{a tady tomu vubec nerozumim, ale prijde mi to relevantni }

				% nekde u psycholingvistiky to s tim kanarkem, zpivanim a mitim kuze a tak

			\subsection{Psycholingvistické hledisko}

				G. A. Miller, který je tvůrcem WordNetu, se po spolupráci s Chomskym na základních kapitolách jeho \textit{Handbook of Mathematical Psychology}\td{cit}, která se věnuje spíše syntaktickému popisu jazyka, se zaměřil společně s Johnsonem-Lairdem na výzkum, jakým způsobem je lexikální materiál uložen v lidském mozku. Tento přístup je označován jako psycholingvistika a jeho počátky jsou spojeny s průzkumem asociací a budování modelu mentálního slovníku člověka. Výchozí myšlenka, jež se odráží i ve způsobu organizace WordNetu, spočívá v tom, že slovní zásoba není organizována abecedně, jako tomu je v tradičních slovnících, ale spíše konceptuálně. 

				Jednou z otázek tohoto směru bylo, jakým způsobem je organizována paměť. Aby člověk byl schopen určit pravdivostní hodnotu výroku \ex{Kanárek může létat}, musí použít svou dlouhodobou pamět. Její organizace je pak možná (minimálně) dvěma způsoby. První, redundantní, by vypadal tak, že by u každého ptáka bylo uloženo, že je schopen létat. Druhý, již na první pohled výrazně méně náročný na úložný prostor, by příznak schopnosti létat měl uložený pouze u kategorie \ex{pták}. V případě, že by bylo třeba zjistit, zda kanárek léta, by bylo nutno pak zapojit inferenční proces ve stylu \textit{kanárek je pták, tudíž může létat}. \parencite{collins1969retrieval}

				Jak \textcite{collins1969retrieval} dále uvádí, lze předpokládat, že v případě prvního způsobu organizace paměti by člověk mohl kteroukoliv informaci o příznacích (vlastnostech) z paměti vyvolat za konstantní čas. Naproti tomu v případě způsobu druhého by extrakce příznaku z konceptu v hierarchii položeného výše měla trvat delší čas než extrakce příznaku přítomného přímo u konceptu, jenž je objektem věty. Důvodem by měla být nutnost zapojení inferenčního procesu.

				Pokus, kterým podpořili \textcite{collins1969retrieval} druhý, neredundantní, způsob organizace paměti, spočíval v tom, že testovací subjekty, dobrovolníci z řad zaměstnanců společnosti Bolt Beranek and Newman, měly určovat, zda je výrok pravdivý, či nepravdivý. Měli tak činit co nejpřesněji a v co nejkratším čase, přičemž byla měřena rychlost jejich reakce. Ukázalo se, reakční doba při určování pravdivosti výroku \ex{Kanárek umí létat}\footnote{angl. \ex{A canary can sing}} je delší než při určování pravdivosti výroku \ex{Kanárek umí zpívat}\footnote{angl. \ex{A canary can sing}} a ještě delší při určování výroku \ex{Kanárek má kůži}\footnote{angl. \ex{A canary has skin}}. Důvodem pro tyto progresivní prodlevy podle nich právě byla zvětšující se vzdálenost od konceptu \ex{kanárka} ke konceptu, u něhož byl uložen příslušný příznak, tedy \ex{umí zpívat}, \ex{umí létat}, resp. \ex{má kůži}. Příznak \ex{umí zpívat} totiž je pravděpodobně uložen přímo u \ex{kanárka}, jelikož jej odlišuje od ostatních ptáků, zatímco příznak \ex{umí létat} je obecným znakem ptáků, tudíž je uložen u konceptu \ex{pták}. V poslední řadě pak příznak \ex{má kůži} bude patrně uložen u konceptu \ex{zvíře}, který je oněch tří v hierarchii nejvýše, a ze všech tudíž od kanárka nejdále.

				Jelikož WordNet G. A. Millera je založen na podobném principu\td{cit}, lze říci jistým způsobem odráží organizaci lexika v lidském mozku.

				Zde je na místě uvést poznámku o dalším pricnipu, o nějž se WordNet opírá, a to organizace slovní zásoby podle základních slovních druhů. Testováním anaforických a komparativních výrazů se ukázalo, že lidé dokážou vcelku jednoduše určit slovní druh určitého výrazu. Je tedy nasnadě, že taková informace musí být přítomna u každého konceptu, což vede, oproti předchozímu principu, k jisté redundantnosti systému. Existuje totiž mnoho slov (zvláště např. v angličtině), která zastupují jak substantivum, tak verbum, a zdá se, že tyto koncepty, byť označovány stejným výrazem (angl. \ex{show}, popř. česky \ex{stát}), jsou uloženy zvlášt a mají svou vlastní množinu příznaků. Tento koncept podporuje i fakt, že se evidentně chovají zcela odlišně v rovině syntaktické.\td{i should cite this shit, ale fakt netusim, kde uz jsem to zas vycetl.. a pulku jsem si stejne vymyslel sam xD}


		
		\section{Historie Wordnetu}




	% \begin{spacing}{1.05}
	\printbibliography[title={Seznam literatury}]
	% \end{spacing}

	\addcontentsline{toc}{chapter}{Seznam literatury}

\end{document}