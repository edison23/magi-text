\documentclass[a4paper, 11pt, oneside]{book}
% \usepackage[inner=3.5cm,outer=2.5cm, top=2.5cm]{geometry}
\usepackage[left=3cm,right=3cm]{geometry}
\usepackage{fontspec}
\usepackage{lmodern}
\usepackage[english,czech]{babel}
\usepackage{csquotes}
\usepackage{url}
\usepackage{todonotes}
\usepackage{xunicode}
\usepackage{graphicx}
\usepackage[pagestyles,medium]{titlesec}
\usepackage{setspace}
% \usepackage{tikz-qtree}
\usepackage{tikz}
\graphicspath{ {imgs/} }
\usepackage{verbatimbox}
\usepackage{nameref}
\usepackage{enumitem}

% todo:
% - reference na kapitoly - itNameRef a ref, vsechno ma byt itNameRef asi?
% todo notes!
% vsechny priklady maji byt \ex{} - nutno kontrolovat hodne v poznamkach pod carou s preklady - \ex{run}\footnote{čes. \ex{běžet}}
		% \ex{} dela vizualne priklad, definovano nize pod nejakym \newcommand{}
% mozna vysvetlit pojmy slovo/slovni forma - vyznam -- prvni dve pouzivam ve vyznamu slovni formy (neceho, co je jen jedno nebo jak to rict), vyznam je ve svem obvious vyznamu
% sjednotit carky za etc., asi tam byt nemaji (podle prirucky a v ni zmineneho atd., apod., ...)
% napsat, ze prace vznikala delsi dobu, takze rozhrani nebyla testovana v jednu chvili a v prubehu byla nektera updatovana?
% zkontrolovat, ze data pisu ve stejnem formatu, tedy asi D. M. YYYY
% ramecky kolem vsech obrazku, jak?
% pisu vsude starym nebo novym pravopisem?

\setmainfont{Vollkorn}
\linespread{1.2}
\setlength\emergencystretch{1em}
\setlength\headheight{14pt}
% \setstretch{1.5}
\setcounter{secnumdepth}{3}
\setcounter{tocdepth}{1}

\widowpenalty10000
\clubpenalty10000

\setlist{noitemsep}

\newpagestyle{sensible}{
	\headrule\sethead{}{}{\MakeUppercase{\chaptertitle}}
	\setfoot{}{\thepage}{}
}

\def\verbarg{\setstretch{0.79}{\scriptsize\makebox[2ex]{\arabic{VerbboxLineNo}}}\hspace{2ex}}

% https://github.com/michal-h21/biblatex-iso690
% unzip to ~/texmf/tex/latex and use as style=iso-*
% some possible styles: iso-authoryear, iso-numeric, iso-alphabetic

\usepackage[
   backend=biber
  ,style=alphabetic
  ,style=iso-alphabetic
  ,sortlocale=cs_CZ
  ,autocite=footnote
  ,maxnames=2
  ,minnames=1
  ,urldate=long
  % ,spacecolon=false
  ]
  {biblatex}

% to be fixed, this is to substitute non-existen czech norm by some other similiar, but croatian is not the one we want (does it matter?)
\DeclareQuoteAlias{croatian}{czech}

% and theres something like this:
% czech quotes
% \usepackage{csquotes}
% \DeclareQuoteStyle{czech}
%   {\quotedblbase}
%   {\textquotedblleft}
%   {\textquoteleft}
%   {\textquoteright}

% \renewcommand{\uv}[1]{
%   \enquote{#1}
% }

\addbibresource{bibliografie.bib}
\renewcommand*{\labelalphaothers}{\textsuperscript{+}}

\newcommand{\td}[2][]{
	{\hskip -0.5em\todo[size=\footnotesize]{#2}}
}
\newcommand{\simplywn}{\textit{NSFW viewer} }

\newcommand{\itNameRef}[1]{\textit{\nameref{#1}}}

% \def\inch#1{#1''}

% \providecommand\parencite{}
% \providecommand\textcite{}

% \renewcommand\parencite{\cite}
% \renewcommand\textcite{\cite}

\author{edison}
\title{Vizualizace sémantické sítě}

\newcommand\ex{\textsf}

\begin{document}
	
	% \pagestyle{empty}
	\begin{titlepage}
		\begin{center}
			{\Large\uppercase{Masarykova univerzita}}

			\vspace{1em}

			\includegraphics[width=0.24\textwidth]{logo-muni.png}
			
			\vspace{2em}

			{\Large Filozofická fakulta}

			\vspace{1em}

			{\large Ústav českého jazyka}

			\vspace{1em}

			{\large Český jazyk se specializací počítačová lingvistika}

			\vspace{7em}

			{\Large David Klement}
			
			\vspace{5em}
			
			{\huge\bf Vizualizace sémantické sítě}

			\vspace{1.5em}

			{\Large Magisterská diplomová práce}

			\vfill
		\end{center}
		\begin{flushleft}
			Vedoucí práce: RNDr. Adam Rambousek, Ph.D. \hfill 2017
		\end{flushleft}
	\end{titlepage}

	\newpage

	{\setstretch{1.12}
		\tableofcontents
	}

	\newpage

	\chapter*{Úvod}\label{uvod}

	\addcontentsline{toc}{chapter}{Úvod}

		Tato práce se zabývá sémantickými sítěmi typu wordnet především z pohledu lidského uživatele. Sémantická síť princetonský WordNet a další sítě z ní vycházející byly projektovány se záměrem napomoci zejména počítačovému zpracování přirozeného jazyka. Díky své struktuře založené na psycholingvistických poznatcích jsou však potenciálně užitečné i pro lidského uživatele. Tento potenciál společně s vědními základy, které se staly formanty struktury wordnetů, práce rozebere ve své části \ref{part:eins}.

		Aby však sémantická síť byla pro lidského uživatele použitelná, je nutné umožnit k ní přístup pomocí rozhraní, jež zájemci zobrazí relevantní data. Takových rozhraní bylo vytvořeno za dobu vývoje sémantických sítí značné množství; vytvořená rozhraní se liší uživateli, na něž jsou zaměřena, platformami, které podporují či kupříkladu způsoby prezentace dat wordnetů. Jejich společným rysem však bývá jistá neaktuálnost návrhu uživatelského rozhraní, a to především v kontextu jednoho z trendů poslední dekády, kterým je přesun uživatelů z klasických stolních počítačů a laptopů na mobilní zařízení  \parencite{grace2013mobile}.

		V části \ref{part:zwei} se práce bude věnovat právě těmto existujícím rozhraním, rozebere jejich funkcionalitu, positiva a negativa a zhodnotí jejich přínos a použitelnost pro koncového uživatele. Hodnocená rozhraní je nutno brát pouze jako zástupce většího množství dostupných rozhraní (a možná ještě většího těch nedostupných), jelikož cílem této práce není představit co nejúplnější přehled existujících rozhraní, ale spíše ukázat menší množství těch relativně snadno dostupných a podrobněji je rozebrat. Cílem této části je identifikovat nejvýraznější nedostatky těchto rozhraní a vyvodit z jejich zhodnocení některé vlastnosti, kterými by se mělo rozhraní z pohledu současnosti moderní vyznačovat. 

		Praktickým cílem této práce pak bylo vytvořit rozhraní nové, jehož návrh by vycházel z empiricky získaných poznatků při hodnocení existujících rozhraní v části \ref{part:zwei}. V rámci práce bylo takové nové rozhraní vytvořeno, přičemž podrobnějšímu popisu výsledku je věnována \ref{part:drei}. část tohoto textu.

	\newpage

	\part{Sémantické sítě}
	\label{part:eins}

		Sémantické sítě, neboli wordnety, jsou lexikální databáze vytvořené s rozličnými záměry, mezi něž patří například i strojová inference informací v počítačovém zpracování přirozeného jazyka. Ve wordnetech jsou slova slučována podle významů do synonymických okruhů a tyto okruhy jsou propojovány sémantickými vztahy \parencite{pala2013vceska, princetonWN}, čímž celek dostává svému označení sémantická síť.

		\chapter{Princetonský WordNet} % (fold)
		\label{cha:princeton_wn}
		
			Princetonský WordNet je prvním wordnet vůbec. Vznikal na univerzitě v Princetonu pod G. A. Millerem od poloviny 80. let 20. století. Vzhledem k tomu, že byl prvním wordnetem, bylo k němu referováno jako k WordNetu, bez přívlastku. Ačkoliv tento stav v podstatě přetrvává dodnes, oproti době jeho vzniku se situace změnila, vzniklo několik dalších wordnetů a nastala tudíž potřeba je rozlišit. V anglickém prostředí se obvykle pojmem wordnet míní ten princetonský a všechny ostatní wordnety mají přívlastek či vlastní jméno. Příkladem nechť jsou evropské wordnety BalkaNet či EuroWordNet\footnote{O jiných wordnetech než tom princetonském více v dalších kapitolách}. Ačkoliv v mezinárodním prostředí je obvyklé přívlastek \uv{princetonský} používat, bude tato práce pracovat s následujícím rozlišením:

			\begin{itemize}
				\item \textit{WordNet} (ve významu princetonského WordNetu)
				\item \textit{wordnet} (v obecném významu sémantické sítě založené na WordNetu)
				\item \textit{BalkaNet} (ve významu konkrétního wordnetu, např. \textit{BalkaNet} či \textit{wordnet EuroWordNet})
			\end{itemize}

			\section{Motivace vzniku}
				Od počátků snah o zpracování přirozeného jazyka\footnote{angl. natural language processing, zkráceně NLP} bylo nutné poskytnout programu data o lexiku ve zpracovávaném textu, ať už ona data byla jakákoliv. Kupříkladu v začátcích strojového překladu se mělo za to, že stačí ekvivalentní dvojice slov ve zdrojovém a cílovém jazyce, později byla implementována i pravidla, jež měla eliminovat problémy s pořádkem slov ve větách, ale výsledné překlady stále nebyly po kvalitativní stránce uspokojivé, neboť vyžadovaly značné množství editací po zpracování automatickým systémem. \parencite{hutchins1982evolution}

				Tradičně se lexikální materiál ukládal způsobem nikoliv diametrálně odlišným od papírových slovníků určených pro lidské uživatele. Ty obvykle obsahují lexikální záznamy seřazené podle abecedy a s potřebnými informacemi o daných slovech, z nichž pak program může čerpat při zpracování textu.

				Jak uvádí \textcite{pala2013vceska}, uspořádání lexikálního materiálu v formě, v jaké je v tradičních slovnících, je sice vhodné pro člověka, jelikož je to v podstatě jediný způsob, jak lidskému uživateli umožnit vyhledání konkrétního slova, zvláště nezná-li jeho význam, pro strojové zpracování ale je ale takto zpracovaný slovník slovník nedostatečný. Seřazení indexu slovních forem je sice pro efektivní vyhledávání slov důležité, struktura tradičního slovníku však neposkytuje vhodně zakódované informace o příbuznosti slov. Jednak kvůli onomu abecednímu řazení inherentně vzdaluje slova, jež člověk chápe jako nějakým způsobem blízká \parencite{pala2013vceska}, jednak neobsahuje dodatečné vazby mezi příbuznými slovy (nebo je obsahuje ve formě nevhodné pro strojové zpracování). Významová blízkost může vyplývat ze vztahu volné synonymie, antonymie, podřazenosti, nadřazenosti, etc. Pokud si tedy například uživatel výkladového slovníku nepříliš obeznámený s daným jazykem vyhledá určité heslo, dozví se sice pravděpodobně jeho význam, ale nebude schopen své znalosti prohlubovat dále kupříkladu zjištěním, jaké slovo odpovídá opačnému významu.

				Dalším všeobecným problémem při využití tradičních slovníků k počítačovému zpracování jazyka je fakt, že lexikografové předpokládají u uživatele slovníku značné encyklopedické znalosti. Zařazují tak do slovníku jen informace dle jejich názoru důležité pro zařazení dané entity do příslušné nadřazené třídy (genus proximum) a rozlišení dané entity v nadřazené třídě (differentia specifica). Vyhledá-li si tedy člověk ve Slovníku spisovného jazyka českého \td{kurva fix, si snad budu muset dojit do knihovny, abych to ocitoval} heslo \ex{vlk}, zjistí následující:

				\bigskip
				\ex{\textbf{vlk: } psovitá šelma šedě (n. šedožlutě) zbarvená, žijící v Evropě, Asii a v Sev. Americe}
				\bigskip

				Definice a priori předpokládá, že uživatel je obeznámen s tím, co je \ex{šelma} a co je \ex{pes}. Pokud takovou znalostí neslyne (což je vcelku představitelné například u cizince), je nucen si tato slova ve slovníku najít a podívat se na jejich definice (pomiňme nyní netriviální úkol převést slovo \ex{psovitá} na základní tvar \ex{pes}). Pokud nerozumí definicím ani nadřazených slov, musí pokračovat v hierarchii dále a dále. 

				Z uvedeného případu plyne, že jakkoliv je možné správným vyhledáváním hyperonym\footnote{nadřazené slovo} dospět k tomu, že \ex{vlk} je konkrétní entita našeho vesmíru, živá bytost o čtyřech končetinách, savec nějakým způsobem přibuzný se psovi, má šedou srst etc., je takový proces dosti komplikovaný. Případ s cizincem se sice nemusí zdát zcela relevantní, protože se dá předpokladat, že daný člověk má, byť v jiném jazyce, stejné základní znalosti předpokládané lexikografy jako člověk, jehož mateřštinou je čeština. Situace je však dramaticky jiná u počítače (přesněji u počítačového programu). Na rozdíl od člověka totiž počítač nemá žádné předchozí znalosti, tudíž musí projít celým procesem popsaným výše, aby byl schopen kupříkladu určit, že \ex{vlk} může umřít (ježto je živá bytost). Protože však tradiční slovníky typu SSJČ byly vytvářené pro papírové médium, neobsahují žádné propojení ve stylu \textit{toto je odkaz na hyperonymum}, a počítač tudíž jen těžko může zjišťovat, na které vlastně slovo se to má podívat, aby se dobral podstaty pojmu \ex{vlk}.

				\subsection{Strojově čitelné slovníky}

					V zájmu automatizace vyhledávání ve slovníku vznikaly tzv. strojově čitelné slovníky\footnote{angl. \textit{machine readable dictionary}}, což je pojem souhrnně označující lexikální databáze. Podle množství informací, které taková databáze obsahuje, pak lze tyto dělit na slovníky, taxonomie a ontologie\td{mam tady tu zminku vubec nechavat, nebo to mam jeste nejak rozvest.. je to duezite, ale nevim, jak moc pro tuhle praci}. Běžně obyčejný slovník neobsahuje oproti tradičnímu papírovému slovníku navíc žádné metainformace, takže je počítač při jeho užívání v podstatě omezen na elektronický listovač \parencite{miller1990introduction}. 

					Míru, jakou se strojově čitelný slovník odliší od pouhé zdigitalizované formy papírového slovníku a přiblíží se k pokročilé lexikální databázi, lze vyjádřit v několika stupních. V případě, že slovník má jednotlivé významy uspořádány v hierarchii dle nadřazenosti--podřazenosti, lze jej označit za taxonomii, tedy systém s hlubší strukturou než pouze abecedním řazením hesel. 

					Dalším stupněm je již komplexní lexikální databáze, která má jednotlivé významy propojeny rozličnými vztahy, počínaje onou základní hyperonymií a hyponymií\footnote{podřazeností} a pokračuje kupříkladu vztahy meronymie\footnote{vztah \textit{je částí}, tedy např. \ex{dveře} je meronymem \ex{trolejbusu}} či antonymie\footnote{protikladu}. Kromě vztahů mezi významy bude taková lexikální databáze obsahovat zřejmě i další informace, například o syntaktických kategoriích slov, definice jejich významů, etc. Databáze tak popsaných významů propojených sémantickými a případně i lexikálními vztahy\footnote{podrobněji v kapitole \ref{cha:lexvztah} na straně \pageref{cha:lexvztah}} může být nazývána ontologií. \parencite{garshol2004metadata}

				\subsection{Od slovníků k WordNetu}

					Výše uvedená opozice papírového slovníku a ontologie ilustruje rozdíly tradičního slovníku a počítačově zpracovatelné lexikální databáze. Jedním z klíčových rozdílu je propojenost jednotek v lexikální databázi -- tradiční slovníky, byvše v době svého vzniku většinou určeny pro distribuci v papírové formě určené pro lidského uživatele, neobsahují důsledné propojení sémanticky souvisejících slov. Příkladem budiž \ex{kostra} a její části, např. \ex{lebka}. V SSČ\footnote{Slovník spisovné češtiny} i SSJČ se u \ex{lebky} uvádí, že jde o \ex{kostru hlavy}. Lze tedy s jistou rezervou tvrdit, že heslo obsahuje své holonymum\footnote{vztah opačný k meronymii; tedy např. \ex{dům} je holonymem pro \ex{okno}, \ex{dveře}, \ex{práh} etc.}, opačný odkaz však již ani jeden z oněch dvou slovníků neobsahuje. Z celkem evidentních ekonomických důvodů nejsou u hesla \ex{kostra} uvedeny všechny její části. Tento příklad příhodně ukazuje i jistou nesystémovost tradičních slovníků, která je pro počítačové zpracování fatální, jelikož, jak bylo zmíněno výše, znemožňuje systémové procházení hierarchie slovní zásoby a zjišťování podstaty jednotlivých významů.

					Naznačeny tedy byly vlastnosti, jež by lexikální databáze měla oproti tradičnímu slovníku mít, aby byla použitelná pro počítačové zpracování přirozeného jazyka. Především jde o systémovost vztahů. Hypero--/hyponymie je vztah oboustranný, tudíž by mělo být možné se stejnou cestou dostat od nadřazeného slova k podřazenému a naopak. Dále je podstatné, aby sémantické vztahy mezi významy byly přesně definované, a tudíž algoritmicky zpracovatelné. Jedině tak je totiž možno jednoznačně určit, které slovo (či slova) je v takové databázi konkrétnímu slovu nadřazené, které je jeho specifikací, označením jeho částí, etc. 

					S touto myšlenkou vznikl WordNet -- lexikální síť významů provázaných sémantickými vztahy a sovních forem provázaných lexikálnámi vztahy tak, aby to dle poznatků psycholingvistiky odráželo organizaci lexikálního materiálu v lidském mozku (více v kap. \itNameRef{cha:psycho} na straně \pageref{cha:psycho}). \parencite{pala2013vceska} 

					% Zde by bylo na místě poznamenat, že ačkoliv se tak z odstavců výše může čtenáři jevit a i všeobecně je to často tvrzeno, WordNet není ontologií v pravém slova smyslu, protože něco něco.. \url{https://en.wikipedia.org/wiki/WordNet#WordNet_as_a_lexical_ontology} \td{a tady tomu vubec nerozumim, ale prijde mi to relevantni }

					% nekde u psycholingvistiky to s tim kanarkem, zpivanim a mitim kuze a tak

			\section{K vlivu psycholingvistiky na organizaci WordNetu}
			\label{cha:psycho}

				Jelikož G. A. Miller, který byl koordinátorem projektu WordNet, byl svým zaměřením psycholog a přispěl k vzniku psycholingvistiky, ubíral se projekt Wordnetu podobným směrem. Společně s Johnson-Lairdem se Miller zaměřil na výzkum, jakým způsobem je lexikální materiál uložen v lidském mozku. Tento vědní směr je označován právě jako psycholingvistika a jeho počátky jsou spojeny s průzkumem asociací a způsobem modelování mentálního slovníku člověka. Výchozí myšlenka, jež se odráží i ve způsobu organizace WordNetu, spočívá v tom, že slovní zásoba je organizována konceptuálně a pro některé slovní druhy (zejména substantiva) hierarchicky. Konceptuálním uspořádáním je míněno seskupování slovních forem s podobným významem, hierarchická organizace pak staví významy do vztahů například zmíněné hyperonymie a hyponymie, tedy nadřazenosti a podřazenosti.

				Jednou z otázek tohoto směru bylo, jakým způsobem je v hierarchickém modelu paměti řešeno získávání vlastností pro význam, které jsou \uv{poděděné} po významech hierarchicky výše umístěných. Aby člověk byl schopen například určit pravdivostní hodnotu výroku \ex{Kanárek může létat}, musí použít svou dlouhodobou pamět. Její organizace je pak možná (minimálně) dvěma způsoby. První, redundantní, by vypadal tak, že by u každé podtřídy ptáků bylo uloženo, že její instance jsou schopny létat. Druhý, již na první pohled výrazně méně náročný na úložný prostor, by příznak schopnosti létat měl uložený pouze u třídy \ex{pták}. Pro zjištění, zda kanárek létá, by pak bylo nutno zapojit inferenční proces ve stylu \textit{kanárek je pták, tudíž může létat}. \parencite{collins1969retrieval}

				\textbf{OBRAZEK?}

				Jak \textcite{collins1969retrieval} dále uvádí, lze předpokládat, že v případě prvního způsobu organizace paměti by člověk mohl kteroukoliv informaci o příznacích (vlastnostech) z paměti vyvolat za konstantní čas. Naproti tomu v případě způsobu druhého by extrakce příznaku z významu v hierarchii položeného dále měla trvat delší čas než extrakce například příznaku přítomného přímo u významu, kvůli němuž celý proces extrakce probíhá. Důvodem by měla být nutnost zapojení inferenčního procesu a jeho délka rostoucí se vzdálenostmi významů cílového a zdrojového, tedy toho, pro nějž je příznak zjišťován, a toho, z něhož jsou znalosti o vlastnostech zdrojového významu čteny.

				Pokus, kterým podpořili \textcite{collins1969retrieval} druhý, neredundantní, způsob ukládání příznaků v paměti, spočíval v tom, že testovací subjekty, dobrovolníci z řad zaměstnanců společnosti Bolt Beranek and Newman, měly určovat, zda jim předložený výrok připadá pravdivý, či nepravdivý. Měli tak činit co nejpřesněji a v co nejkratším čase, přičemž byla měřena rychlost jejich reakce. Ukázalo se, reakční doba při určování pravdivosti výroku \ex{Kanárek umí létat}\footnote{angl. \ex{A canary can fly}} je delší než při určování pravdivosti výroku \ex{Kanárek umí zpívat}\td{check, jestli to nebylo umi mluvit, takhle to nedava moc smysl..}\footnote{angl. \ex{A canary can sing}} a ještě delší při určování výroku \ex{Kanárek má kůži}\footnote{angl. \ex{A canary has skin}}. Důvodem pro tyto progresivní prodlevy podle nich byla právě zvětšující se vzdálenost od významu \ex{kanárka} k významu, u něhož byl uložen příslušný příznak, tedy \ex{umí zpívat}, \ex{umí létat}, resp. \ex{má kůži}. Příznak \ex{umí zpívat} totiž je pravděpodobně uložen přímo u \ex{kanárka}, jelikož jej odlišuje od ostatních ptáků, zatímco příznak \ex{umí létat} je obecným znakem ptáků, tudíž je uložen u významu \ex{pták}. V poslední řadě pak příznak \ex{má kůži} bude patrně uložen u významu \ex{zvíře}, který je z oněch tří v hierarchii nejvýše, a tudíž od významu \ex{kanárek} nejdále.

				WordNet se svou hierarchickou organizací slovní zásoby tedy pravděpodobně konceptuálně blíží organizaci lexika v lidské paměti.

				Navzdory tomu, že je WordNet (a všeobecně wordnety) využíván převážně k úkolům počítačového zpracování jazyka, zdá se být pro svou podobnost v uspořádání informací vhodný i v jiných oblastech. V souladu s příklady, jež byly předloženy výše, se zdá, že zkoumání významů a propojenosti konceptů v jazyce, s nímž není uživatel zcela obeznámen, by mohlo ve WordNetu být snazší než v klasickém slovníku. V podobném duchu využili při svých kursech WordNet například \textcite{lemnitzer2003using}.
			

			\section{Organizace WordNetu}

				Ve WordNetu lze nalézt informace autosémantikách, slovech plnovýznamových, tedy substantivech, adjektivech, slovesech a příslovcích \parencite{vossen1998introduction}. Synsémantika (např. předložky, spojky etc.) nebyla zahrnuta, jelikož se zdá, že jsou uložena odděleně od slov plnovýznamových. Teorii, že jsou funkční slova uchovávána jako součást syntaktikonu, podpořil kupříkladu \textcite{garrett1982production} při svém pozorování afatických pacientů. 

				Vůbec první podnět k uvědomění, že různé slovní druhy podléhají různé strukturalizaci v paměti, vyvolal asociační test, který provedli \textcite{fillenbaum1965grammatical}. Tomuto asociačnímu testu byly podrobeny anglicky mluvící subjekty, které měly za úkol uvést první slovo, které je napadne při myšlence na předložené slovo. Předkládána jim byla dobře známá a často používaná slova náležející k různým slovním druhům. Ukázalo se, že ve většině případů náleží asociované slovo ke stejnému slovnímu druhu jako slovo, které asociaci vyvolalo. Substantiva vyvolala asociaci na substantivum v 79 \% případů, adjektiva v 65 \% případů a slovesa v 43 \% případů.\td{zkontrolovat procenta, nesedi. spis to melo byt, adj->adj, apod.} 

				Ačkoliv není zřejmé, jak je znalost o slovním druhu určitého slova získávána, lze z uvedených dat předpokládat, že slovní druh je vskutku primární organizační vlastností lexikálního materiálu v lidském mozku a informace o něm je snadno dostupná (alespoň intuitivně). Jelikož správné tvoření vět vyžaduje minimálně podvědomé ponětí o tom, které slovo náleží do které syntaktické kategorie, není s podivem, že tato informace je dostupná lidskému uvažování velmi jednoduše. Jelikož se však slova stejného slovního druhu příliš často nevyskytují pohromadě, není evidentní, jak tyto znalosti člověk získává. \parencite{fillenbaum1965grammatical, miller1990introduction}

				\subsection{Synsety a vztahy mezi nimi}
				\label{cha:princeton-synset-rels}

					% do vztahu by slo zaradit neco o vztazich obecne - Semantic Relations and the Lexicon (M. Lynne Murphy) - 

					Slova (slovní formy) jsou ve WordNetu seskupována podle svého významu (a dále také slovního druhu), k němuž náležejí. Řadě slov přináležejících ke stejnému významu se v terminologii WordNetu říká synset \footnote{angl. zkrácenina pro \textit{synonym set}}, neboli synonymická řada. Každý synset reprezentuje jeden význam, ale je nutno mít na paměti, že granularita synsetů nemusí být konsistentní a v podstatě záleží na tom, jak si tvůrci zadefinovali synonymum (více v kap. \itNameRef{cha:synon} na straně \pageref{cha:synon}). Synset je ve WordNetu reperezentací významu a je definován slovy (formami), které obsahuje. Jelikož význam slov je definován tím, v jakém synsetu se vyskytují (ke kterému konceptu náleží), jde v podstatě o kruhovou definici, a tudíž je zřejmé, že definice významů musí být rozšířena. Lze říci, že význam konceptu reprezentovaného synsetem je založen na jeho pozici v celé struktuře. Význam konceptu je tedy definován jeho kontextem, to znamená nadřazenými a podřazenými koncepty. \parencite{kamps2002visualizing} Vztahy mezi koncepty jsou vztahy sémantické, jelikož se týkají významů slov (cf. lexikální vztahy níže v kapitole \itNameRef{cha:lexvztah} na straně \pageref{cha:lexvztah}). 

					Aby bylo možno WordNet použít k inferenčnímu vyvozování závěrů (získávání informací) o slovech strojově, jsou synsety ve WordNetu propojeny vztahy, z nichž je zřejmé, jakou informaci inferenční stroj získá, přejde-li po onom vztahu k dalšímu konceptu. Strojové zpracování textu, potažmo strojová inference informací, se od té lidské liší v jednom zásadním aspektu, kterým je fakt, že stroj nemá k dispozici encyklopedické znalosti o světě, jež má k dispozici člověk. Pro strojové zpracování textu je tedy nutné zajistit, aby všechna potřebná data bylo možné získat z lexikální databáze, v tomto případě z WordNetu.

					Zmíněné kritérium, že slovní formy jednoho synsetu musí náležet k jedné syntaktické kategorii (slovnímu druhu), je podloženo jednoduchým závěrem o nezaměnitelnosti slov přináležejících různým slovním druhům. % Je evidentní, že ať by forma vyjadřující kupříkladu sloveso i substantivum (cf. angl. \ex{run} (\ex{běh} i \ex{běžet})) měla v oněch dvou instancích sebepodobnější význam, jejich záměna by vedla k negramatické větě (více v kap. \itNameRef{cha:synon} na straně \pageref{cha:synon}). \parencite{miller1990introduction}
					Seskupování konceptů podle slovního druhu vede, zřejmě navzdory snaze o ekonomii ukládání informací, kterou se lidský mozek vyznačuje, k jisté redundantnosti systému. Existuje totiž mnoho slov (zvláště např. v angličtině), která zastupují jak substantivum, tak verbum (např. angl. \ex{show}, popř. české \ex{stát}). Míra sémantické podobnosti takových slov může být značně odlišná. V angličtině je relativně běžné, že substantivum popisuje činnost, k jejímuž dějovému vyjádření se užívá sloveso stejné formy (např. \ex{run} vyjadřuje \ex{běh} i \ex{běžet}). U zmíněného českého \ex{stát} sice lze vypozorovat poněkud vzdálenou sémantickou příbuznost (pojmenování pro stát jako organizovanou územní a politickou mocenskou jednotku \parencite{Dorling2003oxforddic} je zřejmě motivováno jako něco stálého, co dlouho \textit{stojí}), ale není to příliš intuitivní a takové dva výrazy nemohou být zařazeny do stejného konceptu. Slova náležející do odlišných syntaktických kategorií se rovněž syntakticky chovají zcela rozdílně a rozhodně v žádném kontextu nemohou být zaměněna jedno za druhé, což také znemožňuje jejich zařazení do stejného synsetu. \parencite{miller1990introduction} 

					Dalším argumentem pro striktní rozdělení slovní zásoby dle slovních druhů je fakt, že různé slovní druhy mají různou hierarchizaci. Jak bude popsáno v kapitole \itNameRef{cha:sem-vztahy} na straně \pageref{cha:sem-vztahy}, například substantiva jsou hierarchizována podle vztahu hyperonymie a hyponymie, přičemž u nich existují další vztahy jako meronymie, která například u sloves existovat nemůže. Naopak vztah antonymie, který je relativně běžný u adjektiv, se u substantiv téměř nevysktuje\footnote{Lze argumentovat, že např. \ex{život} je antonymem pro \ex{smrt}, faktem ale je, že jde o velmi volnou antonymii -- život popisuje stav či průběh doby, kdy je bytost živá, smrt referuje pouze k okamžiku, kdy se z živé bytosti stává mrtvá bytost, tedy rozhodně nejde o přímý protiklad jako například u adjektiv \ex{světlý:tmavý} nebo \ex{špatný:dobrý}. Stejně tak např. \ex{Bůh} a \ex{Ďábel} jsou sice proti sobě pokládané bytosti, ale jejich antonymie spočívá spíše ve vlastnostech jim připisovaných, tedy subjektivních, a uživatel jazyka může prohlásit, že obě tyto bytosti jsou špatné, čímž ztratí svou protikladnost.}. Verba jsou zase provázána vztahy vyplývání, který u substantiv není příliš evidentní a intuitivní\footnote{Asi lze tvrdit, že z \ex{života} vyplývá \ex{smrt}, ale pravděpodobně takto provázaných substantiv nebude mnoho.}, ale u sloves je vcelku hojný -- například z činnosti \ex{zírat} vyplývá i \textit{nadřazená} činnost \ex{hledět}.

					Sémantické relace mezi slovy různých kategorií ve WordNetu neexistují, avšak pro tyto případy jsou definovány relace lexikální. Oproti sémantickým relacím, které provazují celé koncepty, jsou lexikální relace definovány na úrovni jednotlivých forem. Dvě stejné formy, například \ex{run}, jedna náležející k substantivům, druhá k verbům, budou propojeny vztahem derivačně příbuzné formy\footnote{angl. \textit{derivationally related form}}. \parencite{wndocsWNgloss}

				% Slovní zásoba je proto ve WordNetu propojena pomocí několika sémantických vztahů, které určují její hierarchizaci a odráží například i vzdálenost konceptů (relevantní kupříkladu pro časovou náročnost vyhodnocení pravdivosti výroku, podr. v kap. \itNameRef{cha:psycho} na straně \pageref{cha:psycho}). Hierarchizaci slovní zásoby ve WordNetu lze naznačit grafem \ref{fig:hierchWN} na straně \pageref{fig:hierchWN} (daco takoveho, ale ne tak krepo nakresleneho). 

				% \begin{figure}[h]
				% 	\centering
				% 	\includegraphics[width=1.0\textwidth]{screenshot_2017-03-31_14-14-36.png}
				% 	\caption{Hierarchizace slovní zásoby ve WordNetu}
				% 	\label{fig:hierchWN}
				% \end{figure}

				% Hierarchická organizace významů substantiv\footnote{a možná i dalších slovních druhů} ve WordNetu tak, jak byla naznačna na grafu \ref{fig:hierchWN} na straně \pageref{fig:hierchWN}, je podpořena i poznatkem, že lidé jsou schopni velice rychle zpracovávat anaforické a kataforické výrazy a komparativní konstrukce. Například ve větě \ex{Vlastnil pušku, ale z té zbraně se nikdy nevystřelilo.} je každému čtenáři zřejmé, že výraz \ex{ta zbraň} odkazuje k výrazu \ex{puška}.\footnote{Zajímavé je uvažovat o významu této věty při vypuštění deiktického \ex{ta} -- zdá se, že pak už anaforický odkaz k \ex{pušce} nefunguje a z věty se stává jakési nesmyslné spojení dvou výroků -- konkrétního (\ex{Vlastnil pušku} a zcela obecného (a evidentně nepravdivého) \ex{ze zbraně se nikdy nevystřelilo}.} Co se zmíněného zpracovávání komparace týče, lze říci, že nelze porovnávat dvě substantiva, která jsou provázána sémantickým vztahem hyperonymie-hyponymie. Výrok \ex{Puška je bezpečnější než zbraň} je zcela nesmyslný. \parencite{pala2013vceska} % nasledujici je asi uplne blbost: \footnote{Porovnání substantiv svázaných vztahem hyperonymie--hyponymie však je možné v případě, že je vypuštěno \ex{než}; specifičtějšímu ze slov se tím pak přisuzuje rys, jenž u hyperonyma není přítomen, či se rys hyperonyma stupňuje: \ex{bytový dům je takový hezčí panelák} či \ex{rys je taková divočejší kočka} (lípa je takový vyšší strom, panelák je takový vyšší dům, ...)}.



			
			% \section{Historie WordNetu}

			% 	- WordNet was created in the Cognitive Science Laboratory of Princeton University under the direction of psychology professor George Armitage Miller starting in 1985
			% 	- been directed in recent years by Christiane Fellbaum
			% 	- George Miller and Christiane Fellbaum were awarded the 2006 Antonio Zampolli Prize
			% 	- As of November 2012 WordNet's latest Online-version is 3.1.

			\section{Sémantické vztahy WordNetu}
			\label{cha:sem-vztahy}

				% Jak bylo naznačeno výše, koncept WordNetu je založen na lexikální sémantice, tedy představě, že slovo je kombinací slovní formy a významu, nebo slovního významu. Slovní forma je projevem \uv{fyzickým}, tedy je to vyřčená či napsaná instance významu. Jak je zjevné z přirozeného jazyka, nelze počítat s tím, že by zobrazení významu na formu bylo bijektivní, tedy každý význam byl namapován jedna ku jedné na slovní formu. V přirozeném jazyce může jedna forma zastupovat více významů a jeden význam může být vyjádřen více formami. Příkladem budiž slovní forma \ex{koruna}, která může zastupovat význam měny, vrcholku stromu, vladařského odznaku, etc. Toto zobrazení jedné formy na více významů se nazývá polysémií nebo homonymií\footnote{obojí znamená totožnost formy pro různé významy, u polysémie však ony významy mají společný základ (byť může být velmi vzdálený)}. S polysémií souvisí ještě homonymie, což ve své podstatě dosti podobný vztah, ale totožnost formy je zcela nahodilá. Kupříkladu formu \ex{kolej} lze interpretovat jako referenci k stopě po voze, případně dvojici kolejnic jako vodící dráze pro dopravní prostředky\td{cit. SSJC} a zároveň jako zařízení vysoké školy pro ubytování studentů.\td{cit SSJC}. U významů formy \ex{koruna} lze vypozorovat nějaký společný základ (koruna stromu je nahoře, panovnickou korunu má panovník na hlavě, tedy nahoře, koruna jako mince zase pravděpodobně získala své pojmenování díky faktu, že na mincích bývá vyobrazen panovník). Naproti obě formy \ex{kolej} pochází z odlišeného základu -- \ex{kolej} jako ubytovací zařízení pochází z latinského \ex{collegium}, kdežto výraz pro dráhu je odvozeno od českého \ex{kolo}\td{cit: etymolog. slovnik, ale jeho online verze to neuvadi}.

				% \textcite{miller1990introduction} popisují výše naznačené vztahy pomocí takzvané lexikální matice. Ta názorně zobrazuje formy synonymní ($F_1$ a $F_2$) a formy polysémní ($F_2$):

				% {\tt tabulka z miller1990introduction pg. 4: http://i.imgur.com/sohtwe5.png}

				% \td{tady by jeste slo pokracovat opisovanim dalsi casti toho clanku (miller1990introduction" pg 5 >>)}

				% V dalších podkapitolách budou rozebrány podrobněji, nikoliv však vyčerpávajícím způsobem, sémentické vztahy konstituující Wordnet. Jelikož se sémantické vztahy pro jednotlivé slovní druhy liší, bude tato kapitola strukturována primárně právě podle slovních druhů a až sekundárně podle sémantických vztahů.

				% proc musi sem. vztahy byt rozdelene podle PoS: o subst. nelze moc rikat, ze jsou antonymni, etc.., 

				% \subsection{Frekvenční distribuce sémantických vztahů ve WordNetu}

					% nejdriv nutno zjistit, jak je to s temi vztahy ruznych slovnich druhu... jinak ta statistika nedava vuebc smysl 

				V této kapitole budou podrobněji rozebrány sémantické vztahy konstituující WordNet. Sémantické vztahy jsou, na rozdíl od vztahů lexikálních, které jsou vztahy mezi slovními formami, vztahy mezi koncepty (významy). Rozdíl nejlépe ilustruje protipříklad. Synonymie je typickým vztahem lexikálním; kdyby byla vztahem sémantickým, znamenalo by to, že dva různé významy (koncepty) mají stejný význam, což je nesmysl, jelikož v tom případě to nejsou dva významy, ale jeden. 

				Struktura těchto vztahů není, jak by se na první pohled mohlo zdát, plochá, ale organizovaná podle syntaktické kategorie významů, jež jsou jimi propojeny. Substantiva mají své vlastní vztahy, stejně tak adjektiva, verba a adverbia. Pojmenování těchto vztahů vychází z lingvistických termínu k nim relevantních (např. hyperonymie), ale v některých případech se pojmenování napříč různými syntaktickými kategoriemi překrývá, ačkoliv jde o vztahy různé. Například angl. sloveso \ex{run}\footnote{čes. \ex{běžet}} má ve WordNetu jako hyperonymum uveden synset s významem \ex{pohybovat se velmi rychle} obsahující slovesa \ex{travel rapidly, speed, hurry, zip}\footnote{čes. \ex{cestovat rychle, uhánět, ...}}. Je evidentní, že tento vztah hyperonymie není identický se vztahem hyperonymie u substantiv, kde \ex{house}\footnote{čes. \ex{dům}} má jako přímé hyperonymum uveden synset \ex{building, edifice}\footnote{čes. \ex{budova, stavba}}. Z činnosti \ex{běžet} vyplývá činnost \ex{rychle se pohybovat}, ale \ex{budova} je pro \ex{dům} nadřazenou třídou. Jde tedy o vztah nikoliv nepodobný, ale ne identický. \parencite{princetonWN}

				Vzhledem k vlastnostem sémantických vztahů, jimiž jsou synsety WordNetu provázány, tvoří tyto orientovaný graf (což je v rozporu s obecně značně rozšířenou vizualizací WordNetu jako stromové struktury). Pokud se v tomto grafu vyskytnout anomálie (chyby), může se stát i cyklickým grafem (pokud by například synset měl za hyperonymum sám sebe). To je však zřejmě ojedinělý jev. \parencite{richens2008anomalies} Zdánlivé cykly ve vizualizaci však nastat mohou, jelikož jeden koncept může mít více hyperonym (více v kapitole \itNameRef{cha:hyperohyp} na straně \pageref{cha:hyperohyp}), která v určitém bodě budou mít opět společné hyperonymum, jelikož má WordNet jeden kořen.

				% Tato kapitola tedy bude primárně organizována podle slovních druhů a sekundárně podle jednotlivých sémantických vztahů.

				
				\subsection{Hyperonymie a hyponymie}
				\label{cha:hyperohyp}

					Vztah nadřazenosti a podřazenosti strukturalizuje především slovní zásobu substantiv vztahem tříd a podtříd. Jde o vztahy transitivní a asymetrické. \parencite{miller1990introduction} Díky této hierarchizaci se lze například vyhnout redundanci ukládání informací v paměti, jelikož příznaky třídy není nutné ukládat u každé podtřídy. Podtřída dědí všechny příznaky své mateřské třídy a přidává minimálně jeden další. Například \ex{tramvaj} je \ex{pouličním kolovým přepravníkem, který jezdí po kolejích a je poháněn elektřinou}\footnote{ve Wordnetu \textit{streetcar, tram, tramcar, trolley, trolley car (a wheeled vehicle that runs on rails and is propelled by electricity)} \parencite{princetonWN}}. Pokud některý ze zděděných příznaků pro podtřídu neplatí, je tento fakt u ní explicitně uložen (podrobněji v kapitole \itNameRef{cha:psycho} na straně \pageref{cha:psycho}). Uspořádání, v němž jsou atributy takto děděny, se nazývá dědičný systém\footnote{angl. \textit{inheritance system}} \parencite{touretzky1986mathematics}.

					Substantiva jsou ve Wordnetu organizována tak, že každý význam má svůj mateřský význam (hyperonymum), kromě jednoho jediného, a tím je \ex{entity}, tedy uměle vytvořený pojem sloužící jako kořen celé sítě. Jeden koncept může mít hyperonymních významů více, například \ex{house} má jako svá hyperonyma uvedeny synsety \ex{(n) dwelling, home, domicile, abode, habitation, dwelling house} a \ex{(n) building, edifice}.

				\subsection{Meronymie a holonymie}

					Meronymie (a k ní komplementární vztah holonymie) jsou, navzdory nepříliš rozšířenému názvosloví, dalším vztahem, jenž je pro uživatele jazyka intuitivní a známý. Jde o vztah \textit{být částí}, potažmo \textit{mít část}. Meronymie je definována tak, že B je holonymem A v případě, že jednou z částí B je A. Meronymie je vztahem stejně jako hyperonymie transitivním a asymetrickým. \parencite{cruse1986lexical} Tento vztah také hierarchizuje lexikum do určitých úrovní, ale na rozdíl od vztahu nadřazenosti, v nemž obvykle jeden nadřazený význam mívá jeden až více podřazených významů, u vztahu části a celku je situace složitější. Je totiž na snadě, že jeden význam může být meronymem mnoha holonymům -- kupříkladu \ex{dveře} mohou být meronymem například u \ex{dům}, \ex{auto}, \ex{šatník}, \ex{počítačová skříň}, etc.\footnote{Ve WordNetu to může být ilustrováno na jednom významu \ex{dveří}, u něhož jsou jako holonymum uvedeny významy \ex{dveřní otvor, dveře, přístup do pokoje, práh} (\textit{doorway, door, room access, threshold}) \parencite{princetonWN}}

					Vztah části a celku je vlastní výhradně substantivům. U sloves se může vyskytovat vztah vyplývání a příčiny.

				% jeste slovesa - entailment a cause (buy:pay, kill:die)


			\section{Lexikální vztahy ve Wordnetu}
			\label{cha:lexvztah}

				\subsection{Synonymie}
				\label{cha:synon}

					Synonymie je základním definičním vztahem pro synsety ve WordNetu. Na praktických aplikacích je tento jev nejlépe pozorovatelný, jelikož při vyhledání konkrétní formy je uživateli obvykle nabídnut výběr z jednotlivých významů dané formy. Aby byly od sebe významy oné formy odlišitelné, nabídka obvykle sestává ze seznamu skupin slovních forem náležejících do nalezených synsetů\footnote{\url{https://www.englishforums.com/English/AdjectiveSatellite/nwzhv/post.htm#1126701}} Kupříkladu při vyhledání slova \ex{kolo} v českém wordnetu tak je uživatel konfrontován s několika skupinami, které obsahují zhruba následující:

					\begin{itemize}
						\item \ex{kolo (1)},
						\item \ex{jízdní kolo (1), bicykl (1), kolo (2)},
						\item \ex{kružnice (1), cívka (1), kolo (3)},
						\item[] etc.\td{citaci cs wordnetu}
					\end{itemize}

					přičemž čísla (zde) v závorce značí index významu dané formy v daném synsetu. \parencite{pala2004building} Reprezentace v různých aplikacích a různých wordnetech se liší (standardem bývá číslo významu psát za dvojtečku), význam však zůstává neměnný. 

					Navzdory zdánlivé jednoduchosti uvedeného konceptu je všeobecnou otázkou, jak synonymii pojímat. Striktní teorie (obvykle připisovaná Leibnizovi) praví, že dvě slova jsou synonymní, pokud se jejich záměnou nikdy nezmění pravdivostní hodnota výroku. Lingvistickou interpretací tohoto poněkud matematicko-logického výroku může pak být, že synonymní dvě slova jsou v případě, že se jejich záměnou nikdy neporuší význam (zhruba ona pravdivostní hodnota) a gramatičnost výroku. Je nasnadě, že takto striktně synonymní slova budou pospolu v jazyce těžko přežívat, jelikož je dokázáno, že jazyk tíhne k ekonomičnosti, která by takovým soužitím dvou slov byla hrubě porušena \parencite{Lotko2003}. Pravděpodobně jedinými obecně uznávanými synonymy jsou obvykle dvojice cizího slova a domácího slova, například \ex{internacionální} a \ex{mezinárodní}. Jejich záměnou se velice pravděpodobně nikdy pravdivostní hodnota výroku nezmění, stejně tak jako jeho gramatičnost. Stále však zůstává ve hře stylistika, která může být podobnou náhradou narušena např. z důvodu toho, že se nové slovo nehodí pro zamýšlenou cílovou skupinu čtenářů či je stylisticky příznakové (cf. \ex{zajímavý} a \ex{interesantní}). Co se tendence k ekonomičnosti jazyka týče, lze předpokládat, že v těchto případech převládá potřeba synonym k eliminaci opakování určitých slov v textu a tím zajištění jeho stylistické uhlazenosti. 

					Volnější interpretace synonymie počítá ještě s kontextem. Dvě slova jsou synonymní, jsou-li bez způsobení škod nahraditelná alespoň ve stejném kontextu. Jako příklad mohou posloužit formy \ex{board} a \ex{plank}\footnote{čes. \ex{prkno} a \ex{fošna}}. V kontextu dřevařství mohou tyto dvě formy pravděpodobně bez problému být nahrazeny jedna za druhou, ovšem v případě, že je forma \ex{board} použita ve významu \ex{comittee}\footnote{čes. \ex{výbor}}, těžko ji lze nahradit formou \ex{plank}, neboť by se věta obsahující takové nahrazení stala zcela nesmyslnou. \parencite{miller1990introduction} V českém kontextu mají zřejmě podobný vztah například formy \ex{rada} a \ex{výbor}. Slovní forma \ex{rada} může sloužit jak ve významu rozhodovacího orgánu, což je zhruba synonymní s formou \ex{výbor}, tak ve významu poučení (např. \ex{upřímná rada přítele}). Opět je evidentní, že záměna forem \ex{rada} a \ex{výbor} v prvním významu slova \ex{rada} (tedy rozhodovací orgán) je přijatelné, ve druhém už nikoliv (\ex{upřímný výbor přítele}). Nutno podotknout, že forma \ex{výbor} je také polysémní a její další významy se nekryjou s významy slova \ex{rada}. \parencite{Havranek1989}

					Bylo by nanejvýš přirozené považovat synonymii za vztah diskrétní, tedy že dvě formy buďto synonymní jsou, či nejsou. Z logického hlediska to nepochybně z již uvedeného vyplývá, ovšem lingvisticko-filosofický náhled výcházející z poznatků reálného jazyka na tuto problematiku nahlíží poněkud odlišně. Synonymie v striktním slova smyslu je velice vzácná. Její volnější interpretace je značně častější, ale také výrazně vágnější -- kontext, v němž dvě formy synonymní jsou, může být velmi široký, či naopak velice úzký. Záměna některých dvojic (či spíše obecně n-tic, volné synonymní řady mohou být vcelku dlouhé -- \ex{textil:1, látka:1, textilie:2, plena:1, tkanina:1} \parencite{pala2004building}) může měnit stylistiku a význam výpovědi více či méně, přičemž ony dvě formy stále dle daných kritérií lze považovat za synonymní. Nelze tedy než vyvodit, že synonymie, minimálně z pohledu přirozeného jazyka, je jevem graduálním, a některé n-tice forem jsou tak svázány silnějším vztahem synonymie než jiné (laicky řečeno jsou \textit{synonymnější} než jiné). \parencite{miller1990introduction}

					Zaměnitelnost forem podporuje ještě jeden koncept, na němž je WordNet postaven, a to fakt, že jednotlivé významy jsou seskupovány podle slovních druhů. Tento systém vede k jisté redundantnosti, jelikož zvláště v syntetických\td{manko, nekecam?} jazycích, jako je kupříkladu angličtina, lze nalézt mnoho případů, kdy identická slovní forma zastupuje více slovních druhů. Významy, které taková slovní forma zastupuje (napříč slovními druhy), mohou být velice blízké, nikdy však nebudou stejné (nelze říci, že význam slovesa \ex{run}\footnote{čes. \ex{běžet}} a substantiva \ex{run}\footnote{čes. \ex{běh}} je identický). Jejich záměnou by se sice nestalo vůbec nic, jelikož čtenář či posluchač textu, v němž taková záměna nastala, by automaticky formu interpretoval ve prospěch správného slovního druhu, avšak pokud by slovní druh byl nějakým způsobem \uv{vynucen} (nechť nyní čtenář pomine úvahy, jakým způsobem lze \textit{vynutit} slovní druh formy), stala by se výpověď zcela negramatickou a nesmyslnou. 

					Jakkoliv to není přímo svázané se synonymií, je na místě\td{ne, neni, ale nechtelo se mi mazat 2k napsanych znaku xD} poznámka o výskytu stejné formy v různých synstetech. Slovo je kombinací slovní formy a významu, nebo slovního významu. Slovní forma je projevem \uv{fyzickým}, tedy je to vyřčená či napsaná instance významu. Jak je zjevné z přirozeného jazyka, nelze počítat s tím, že by zobrazení významu na formu bylo bijektivní, tedy každý význam byl namapován jedna ku jedné na slovní formu. V přirozeném jazyce může jedna forma zastupovat více významů a jeden význam může být vyjádřen více formami. Příkladem budiž slovní forma \ex{koruna}, která může zastupovat význam měny, vrcholku stromu, vladařského odznaku, etc. Druhým příkladem může být forma \ex{kolej}, již lze interpretovat jako referenci ke stopě po voze, případně dvojici kolejnic sloužících jako vodící dráha pro dopravní prostředky a zároveň jako zařízení vysoké školy pro ubytování studentů. \parencite{Havranek1989} Toto zobrazení jedné formy na více významů se nazývá polysémií nebo homonymií. (Obojí znamená totožnost formy pro různé významy, u polysémie však ony významy mají společný základ (byť může být velmi vzdálený), u homonymie je podobnost zcela nahodilá. \parencite{klepousniotou2002processing}) Uvedené příklady poslouží i k ilustraci rozdílu mezi homonymií a polysémií. U významů formy \ex{koruna} lze vypozorovat nějaký společný prvek v tom, že bývají nahoře. \ex{Koruna} stromu je nahoře, panovnickou \ex{korunu} má panovník na hlavě, tedy nahoře, \ex{koruna} jako mince zase pravděpodobně získala své pojmenování díky faktu, že na mincích bývá vyobrazen panovník s \ex{korunou} na hlavě). Formy \ex{kolej} jsou odlišného základu -- \ex{kolej} jako ubytovací zařízení pochází z latinského \ex{collegium}, kdežto výraz pro dráhu je odvozen od českého \ex{kolo}. \parencite{Rejzek2012}

					Seskupování významů podle slovních druhů a seskupování forem dle vztahu synonymie se tedy zdá v případě lexikální databáze určené pro strojové zpracování jazyka jako vhodným konceptem. Oproti tradičním slovníkům se totiž počítačově zpracovávaná lexikální databáze nemusí potýkat s problémem lidského faktoru -- jednotlivé synonymické řady je stroj schopen prohledávat, na rozdíl od člověka, vcelku účinně, a nahradí tak v případě, že WordNet používá člověk, neúčinné lidské procházení restříkového obsahu.

					% centralni jednotka WN, davalo by smysl rozlisovat syn. mezi formami a mezi vyznamy, ale pro technickou jednoduchost se to nedela; je to symetricka relace a pokud jsou ji spojeny vyznamy, pak i jejich formy; zamenitelnost - silna: zamenou se nemeni pravdivost vyroku, slabsi: podle kontextu - vyznam se v kontextu zamenou nezmeni (plank × board); z toho plyne nutnost redundance slovnich druhu (show a show nelze zamenit); z log. hlediska diskretni (bud slova syn. jsou ci ne), ale z lingv./fil. hlediska gradient - nektere dvojice jsou syn~ctejsi nez jine- porad je to ale reflexivni; 
					
				% slovesa nemaji meronymii, ale has_a relation -- entailment

				\subsection{Antonymie}
					% problem s terminologii -- co jeste je antonymum, a co uz ne (muz-zena?)
					% pouziti ve slovnicich velmi siroke
					% stupnovatelnost, neutralni prostor na skale, vs. nestupnovatelna adj. - komplementarni, X entails not Y, vs. red-green
					% miller1990introduction tvrdi, ze to neni semanticky, ale lexikalni vztah

					Antonymie, neboli protiklad, je navzdory zdánlivé triviálnosti koncept překvapivě těžce definovatelný. Všeobecně se antonymií rozumí významová opozice, faktem však je, že použití tohoto termínu je velmi široké a druhů antonymie je několik. Nejjednodušším druhem je například antonymie mezi adjektivy \ex{živý} a \ex{mrtvý}. Negace prvého automaticky značí druhé a naopak (je-li řeč o živých bytostech), jelikož v reálném světě neexistuje žádný další třetí stav. Tento jednoduchý vztah však nefunguje vždy -- například s adjektivy \ex{bohatý} a \ex{chudý} je to jiné. Mnoho lidí se nepovažuje ani za chudé, ani za bohaté, a tudíž z toho, že někdo není bohatý, automaticky neplyne to, že by byl chudý. \parencite{miller1990introduction} Zajímavé je, že tento vztah není reflexivní. Pokud někdo není bohatý, tak to nemusí znamenat, že je chudý, ale pokud u někoho platí, že \textit{je} bohatý, tak to nutně znamená, že \textit{není} chudý minimálně v tom ohledu, v němž je bohatý. \parencite{paradis2006antonymy} 

					Rozdíl mezi výše uvedenými dvojicemi, tedy \ex{mrtvý:živý} a \ex{chudý:bohatý} spočívá ve stupňovatelnosti daných adjektiv. Pro ilustraci -- lze říci, že někdo je \ex{bohatší} než někdo jiný, ale nelze říci, že někdo je \ex{\textit{mrtvější}} než někdo jiný. Pokud jsou adjektiva stupňovatelná, tedy lze říci, že objekt A je více X než objekt B, neoznačují komplementární stav, ale graduální vlastnost. Označované pak může být zařazeno kamkoliv mezi tyto dva póly, přičemž nachází-li se v pomyslné střední šedé zóně, nelze jej označit výrazy odpovídajícími pólům gradientu. Tvrzení, že někdo \ex{není ani chudý, ani bohatý}, dává smysl, protože tato adjektiva označují extrémní stavy, mezi nimiž je prostor pro normální stav. \parencite{paradis2006antonymy} 

					Vztah antonymie ve WordNetu je koncipován tak, aby zřejmě byl co nejpodobnější uvažování široké populace uživatelů jazyka, tedy užívá primitivního konceptu antonymie. Některé studie dokonce za antonymní považují výrazy pouze vágně, intuitivně protikladné, jako například \ex{muž:žena} či \ex{chytrý:hloupý}. \parencite{lehrer1982antonymy}

					% taky dopsat neco o tom, ze antonyma jsou si zaroven nejblizsi a zaroven nejvzdalenejsi (lisi se jednim priznakem a jsou na opacnych polech) -- paradis2006antonymy

					% Antonymie podle \textit{miller1990introduction} navíc ani není sémantickým vztahem, ale lexikálním, .. a tuhle poznamku bych si nechal na konec, jelikoz to celkem zabiji.

					Ve WordNetu se antonymie vyskytuje u substantiv (\ex{man:woman}), adjektiv (\ex{rich:poor}, a dokonce i \ex{white:black} v rasovém významu\footnote{cf. také antonymní vztah \ex{Caucassian:black} ve WN}), verb (\ex{open:close}) i adverbií (\ex{well:ill}). \parencite{princetonWN}


					% tak jaky tam vlastne jsou vztahy? jsou deleny pro PoS, ci nikoliv?
		% chapter princeton_wn (end)

		\chapter{Další wordnety}
		\label{cha:dalsi_wordnety}

			% base concept 
				% - koncept, kterej je co nejvys v sem. hierarchii a ma spoustu vztahu na dalsi koncepty (ale proc?)
				% - universalita: 
		
			Podle vzoru princetonského WordNetu začaly postupně vznikat i další sémantické sítě založené na stejném konceptu. Tyto sémantické sítě se samozřejmě svou strukturou do větší či menší míry liší, hlavním kritériem pro to, aby mohly být považovány za wordnet, je to, aby obsahovaly synsety a hyponyma. \parencite{gwa2013wordnetsworld} % Jelikož se tato práce bude primárně zabývat wordnetem českým, bude pro srovnání uvádět dva hlavní evropské vícejazyčené wordnety, a to EuroWordNet a BalkaNet. 

			\section{EuroWordNet} % (fold)
			\label{sec:eurowordnet}
				
				% motivace - aby bylo mozny delat nfo retrieval, search queries augment
				% prolinkovani s PWN1.5 pomoci ILI - interlingual index
					% plochy, aby nereflektoval organizaci nejakeho konkretniho WN
				% odlisne vztahy od PWN

				EuroWordNet je mezinárodní lexikální databáze pro sedm evropských jazyků (angličtina, čeština, dánština, francouzština, italština, němčina, španělština). Jde o soubor jednotlivých národních wordnetů, které jsou propojeny takzvaným mezijazykovým indexem (ILI, \textit{inter-lingual-index}). Obecně jsou wordnety EuroWordNetu založené svou strukturou na princetonském WordNetu (verze 1.5), ale z důvodu různorodosti jazyků se v některých aspektech od něj odlišují. EuroWordNet zavádí poněkud odlišené vztahy, navíc je diferencuje jemněji. \parencite{pazienza2008bottom}

				Základní motivací pro vznik EuroWordNetu byla evropská jazyková různorodost a z ní pramenící problémy ve zpracování dat a napomáhání uživateli v přístupu k neanglickým datům. \textcite{vossen1997eurowordnet} argumentuje, že uživatel musí umět anglicky a být obeznámen s tím, jak je zdroj, v němž vyhledává, napsán, aby byl schopen v něm účinně hledat. Vytvořením wordnetů pro jiné jazyky si slibuje, že se zlepší možnost přístupu uživatelů k neanglickým datům, možnosti inference znalostí z těchto dat a případně i mezijazykové vyhledávání. Poslední je založeno na faktu, že od počátku byly jednotlivé wordnety EuroWordNetu vytvářeny s tím, že budou propojeny na základě základních konceptů (BCS, \textit{Base Concept Sets}) a mezijazykového indexu.

				Jelikož se jednotlivé jazyky zapojené v projektu EuroWordNetu značně odlišují ve struktuře své slovní zásoby, jsou jednotlivé wordnety nezávislé. To znamená, že se mohou odlišovat například svou hierarchizací. Stejný koncept tak může ve dvou různých wordnetech mít různá hyperonyma, meronyma, etc., protože například anglické označení pro \ex{prst} je odlišené, pokud jde o \ex{prst na noze} (angl. \ex{toe}), či o prst na ruce (angl. \ex{finger}). Podobně má v jiném příkladu dánština odlišené označení \ex{hlavy u zvířat}, tedy dán. \ex{kof}, a hlavy lidské (dán. \ex{hoofd}). \parencite{vossen1997eurowordnet}

				Národní wordnety jsou vzájemně propojené přes mezijazykový index s anglickým wordnetem, který je obsahově založený na princetonském WordNetu, ale není identický. Anglický wordnet byl přizpůsoben strukturně tak, aby byl použitelný v EuroWordNetu, tedy byly přidány dodatečné metainformace a druhy vztahů (podrobněji dále). V národních wordnetech existuje několik druhů konceptů, které jsou rozlišeny podle příbuznosti s koncepty v ostatních národních wordnetech. Pokud je koncept přítomen ve všech wordnetech EuroWordNetu, jde o koncept tzv. globální koncept\footnote{Global Base Concept (GBC)}. Koncept, jenž jen přítomen v alespoň dvou národních wordnetech, je označován jako obecný koncept\footnote{Common Base Concept (CBC)} a v poslední řadě koncept, který se vyskytuje pouze v jednom národním wordnetu, nese označení lokální koncept\footnote{Local Base Concept (LBC)}. \parencite{gwa2013baseconcepts} Propojení konceptů společných pro více jazyků je zajištěno pomocí jednotných identifikátorů a mezijazykového indexu, který je nadmnožinou všech konceptů v EuroWordNetu. ILI je hierarchicky plochá struktura (proto \textit{index}, nejde o další \uv{všejazykový} wordnet). \parencite{vossen1997eurowordnet}

				Jelikož v době, kdy EuroWordNet vznikal, byl princetonský WordNet poněkud omezený mimo jiné co se vztahů sémantických týče, vznikly pro EuroWordNet speciální vztahy umožňující úplnější práci s významy. Základní vztahy přejaté z princetonského WordNetu 1.5 jsou uvedeny v tabulce \ref{tab:wn-rels} na straně \pageref{tab:wn-rels}.

				\begin{table}[t]
					\centering
					\label{tab:wn-rels}
					\begin{tabular}{l l l}
					Relace                           & Slovnědruhové kombinace & Příklad                          \\\hline
					antonymie                        & A-A, V-V                & \ex{open:close}                  \\\hline
					hyponymie                        & N-N, V-V                & \ex{car:vehicle}, \ex{walk:move} \\\hline
					meronymie                        & N-N                     & \ex{head:nose}                   \\\hline
					vyplývání\footnote{entailment}   & V-V                     & \ex{buy:pay}                     \\\hline
					následek                         & V-V                     & \ex{kill:die}               
					\end{tabular}
					\caption{Vztahy přejaté z princetonského WN (N: substantivum, A: adjektivum, V: verbum)}
				\end{table}

				Navíc k těmto vztahům byly přidány štítky (\textit{labels}), jež relaci konkretizují. Byly použity následující štítky:

				\begin{itemize}
					\item conjunction/disjunction
					\item non-factive
					\item reversed
					\item negation
				\end{itemize}

				Použití konjunktivního a disjunktivního štítku spočívá v myšlence, že například u meronymie by bylo vhodné rozlišovat, zda jde o části, které dohromady tvoří celek, nebo jde o podčásti částí (např. \ex{nůž} má meronyma \ex{čepel}, \ex{rukojeť}, \ex{ostří}, ale \ex{ostří} je ve skutečnosti meronymem až \ex{rukejeti}, nikoliv přímo samotného \ex{nože}).

				Štítek \textit{non-factive} je používán u kauzální relace, která nemusí být nutně naplněna:

					% \begin{itemize}
					% 	\item \ex{zabít}\hskip 3em \textit{vyústí v}\hskip 3em \ex{zemřít}
					% 	\item \ex{hledat}\hskip 3em \textit{vyústí v}\hskip 3em \ex{najít} \texttt{non-factive}
					% \end{itemize}

					\begin{table}[h]
					\centering
					% \caption{My caption}
					\label{tab:wnlabels}
					\begin{tabular}{lllcc}
						příčina	& vztah & následek & non-factive? & nutně vyplývá? \\ \hline
						\ex{zabít}  & vyústí v & \ex{zemřít} & – & ano                       \\
						\ex{hledat} & vyústí v & \ex{najít} & \texttt{non-factive} & ne
					\end{tabular}
					\end{table}\td{tahle tabulka by byla vhodna ke zkrasneni tak, aby bylo jasne, ze "nutne vyplyva" je muj vysvetlujici dodatek ke stitku non-factive (sedive?)}

				Podobně lze upřesnit pomocí štítků další relace tak, že jsou ve výsledku jednoznačnější a wordnet, v němž jsou takto označené vztahy obsaženy, může poskytovat více informací. Jejich použití v praxi je však značně omezené. \parencite{rambousek2017Ustni}

				% prostor pro rozsireni

				Jako další rozšíření oproti tehdejší verzi princetonského WordNetu přinesl EuroWordNet také relace mezi slovními formami, jež ve stejném tvaru náleží k více slovním druhům, a rozlišení mezi úzkým a širokým vztahem synonymie a antonymie (\ex{near\_synonym}, \ex{near\_antonym}). Argument pro zavedení mezislovnědruhových relací je relativně přímočarý, a to, že umožňují \uv{sblížit} koncepty, které jsou si příbuzné, jen náleží k jinému slovnímu druhu. Nutno podotknout, že v době vzniku této práce je princetonský WordNet ve verzi 3.1 a obsahuje už relaci \ex{derivationally related form}, která zajišťuje přesně toto propojení (více o synsetech v princetonském WordNetu v kapitole \itNameRef{cha:princeton-synset-rels} na straně \pageref{cha:princeton-synset-rels}). 

				Vztahy blízké synonymie a blízké antonymie byly zavedeny pro možnost vyjádření volnější sémantické příbuznosti. \parencite{pazienza2008bottom} Co se vztahu blízkého synonyma  týče, důvodem pro jeho zavedení byl údajně zájem mít možnost přiblížit koncepty, které jsou si významově podobné, ale pouze na své úrovni. U takových konceptů platí, že byť jejich význam je podobný, jejich hyponyma nelze zařadit pod jeden koncept, jelikož se rozdíl mezi oněmi koncepty svázanými vztahem blízké synonymie prohlubuje. Příkladem budiž trojice nizozemských slov \ex{aparaat}, \ex{werktuig} a \ex{instrument}, jež jsou si významově nepříliš vzdálená:

				% \Tree[.voorwerp\\objekt 
				% 		[.lichaam\\tělo]
				% 		[.aparaat\\přístroj]
				% 		[.werktuig\\nástroj][.instrument\\pomůcka]
				% 	]

				\td{prekreslit!}
				\begin{figure}[h]
					\centering
					\includegraphics[width=1.0\textwidth]{stromcik-nlNL.png}
					\caption{Blízká synonyma (k překreslení)}
					\label{fig:near-synon}
				\end{figure}

				Jak je z obrázku \ref{fig:near-synon} na straně \pageref{fig:near-synon} zřejmé, všechna hyponyma slova \ex{voorwerp} (\ex{objekt}) jsou si rovna, avšak některá jsou si rovnější. Tři výše zmíněná slova jsou si navzájem významově výrazně bližší, než jsou si blízká s ostatními hyponymy na jejich úrovni. Právě aby bylo možno tento vztah reflektovat a tím docílit možnosti například nahrazovat za sebe slova, která sice nemohou být ve stejném synsetu, ale jsou si podobná, byl zaveden vztah blízkého synonyma. Lze totiž předpokládat, že uživatel jazyka podobná slova také může zaměnit. \parencite{vossen1997eurowordnet}

				% cely Vossen-Eurowordnet.pdf
				
				% neco psat o BalkaNetu?

	\part{Přehled a porovnání existujících vizualizací sémantických sítí}
	\label{part:zwei}

		Tato část textu bude zaměřena na zmapování existujících dostupných rozhraní a jejich zhodnocení. Výčet jednotlivých rozhraní rozhodně není vyčerpávající, jelikož některé vědecké instituce mohou vyvíjet své nástroje pro práci se sémantickými sítěmi neveřejně, některé nástroje už nejsou dostupné, etc. Nástroje pro vizualizaci\footnote{Terminologická poznámka: v této práci je vizualizací dat wordnetů míněna jak textová reprezentace, tak v nějaké formě grafická.} dat wordnetů byly vybírány pro přehled v této práci podle několika kritérií relevantních především pro koncového uživatele, avšak i pro případné vývojáře dalších wordnetů (tedy zda dané rozhraní mohou použít pro svá data).

		\chapter{Metodologie porovnání}

			\section{Výběr rozhraní}

				Hlavním kritériem pro zařazení do rozhraní do výběru byla jeho dohledatelnost, ať již podle seznamu příbuzných projektů na oficiálních webových stránkách princetonského WordNetu\footnote{\url{https://wordnet.princeton.edu/}} (jenž je, dlužno podotknout, velice neaktuální v tom smyslu, že uvádí velké množství rozhraní, která již nejsou dostupná) či podle dotazu na vyhledávači Google\footnote{\url{https://www.google.com}}:

					\begin{itemize}
						\item wordnet visualization
						\item wordnet visualisation
						\item wordnet graph
					\end{itemize}

				% mozna vypsat vysledky a zaradit ten seznam sem? plus nejak exaktneji algoritmus filtrovani?
				Z přehledu byly samozřejmě vyřazeny implementace, které k době hledání\footnote{duben 2017} už nebyly dostupné. Také nebyla zahrnuta rozhraní, která jsou funkčně podobná jiným. Ačkoliv většina existujících rozhraní zřejmě pracuje buďto výhradně s princetonským WordNetem, či jej jako zdroj dat nabízí jako jednu z možností, nebyl zdroj dat brán v potaz. Do přehledu byla zahrnována především rozhraní, která jsou dostupná z webového prohlížeče, jelikož je to pro koncového uživatele nejpohodlnější způsob přístupu k aplikaci (není nucen instalovat přídavné programové vybavení na svůj počítač kromě webového prohlížeče, který je vcelku rozšířeným vybavením). Jelikož však i rozhraní určená pro klasické počítače\footnote{Terminologická poznámka: klasickým počítačem je myšleno všeobecně zařízení typu stolních počítačů, laptopů, případně tabletů etc., na nichž je provozován operační systém určený pro klasické počítače.} jsou hojně rozšířena a mohou také poskytnout cennou inspiraci i při vývoji webového rozhraní, byla některá z nich zahrnuta též. Jsou však popisována pouze povrchněji a měla by při interpretaci hodnocení být automaticky značně penalizována, jelikož je uživatel mající zájem takové rozhraní použít nucen instalovat například aplikační rozhraní Java, pokud jej nemá. Dalším důvodem k penalizaci je také fakt, tyto aplikace nejsou přímo spustitelné na mobilních zařízeních\td{dodelat zdroj z toho url v footnote}\footnote{Pro Android existuje několik cest, jak aplikaci napsanou v Javě spustit (\url{http://www.wikihow.com/Get-Java-on-Android}), lze však předpokládat, že použitelnost vizualizací wordnetu bude na zařízení s malou obrazovkou poněkud omezená.} \parencite{gronli2014mobile} a jejich portabilita do formy nativních aplikací může omezená a obtížná.

				Podstatná z hlediska hodnocení je také univerzalita, tedy zda je zdrojový kód rozhraní otevřený a je možné jej použít i pro vizualizaci jiné sémantické sítě, než pro kterou bylo rozhraní vyvinuto. Je ovšem dlužno podotknout, že tato informace bývá často nedostupná; potom se v rámci této práce předpokládá, že kód otevřený není.\td{je to tak ok?}

				Zmiňované hodnocení záměrně není vyčísleno exaktně, jelikož každý uživatel má jiné nároky a tato práce si tedy klade za cíl pouze co nejobjektivněji zhodnotit vlastnosti jednotlivých rozhraní, nikoliv čtenáři vnuknout, že jedno rozhraní je lepší než jiné; to musí případně zhodnotit sám na základě předkládaných faktů a jeho vlastních požadavků na rozhraní.

				Do přehledu byla tedy zahrnuta následující rozhraní:

					\begin{itemize}
						\item An interactive visualization of the Princeton WordNet database \par\hspace{2em}{\small \url{http://mateogianolio.com/wordnet-visualization/}}
						\item Artha \par\hspace{2em}{\small \url{http://artha.sourceforge.net/wiki/index.php/Home}}
						\item BabelNet \par\hspace{2em}{\small \url{http://babelnet.org}}
						\item Cornetto Demo \par\hspace{2em}{\small \url{http://cornetto.clarin.inl.nl/wordnet.xql}}
						\item GoldenDict \par\hspace{2em}{\small \url{http://goldendict.org/}}
						\item sloWTool \par\hspace{2em}{\small \url{http://nl.ijs.si/slowtool/}}
						\item Treebolic \par\hspace{2em}{\small \url{http://treebolic.sourceforge.net/}}
						\item Visual Browser \par\hspace{2em}{\small \url{https://nlp.fi.muni.cz/projekty/visualbrowser/}}
						\item wnbroswer\footnote{sic erat scriptum} \par\hspace{2em}{\small \url{http://homepages.inf.ed.ac.uk/adubey/software/wnbrowser/index.html}}
						\item WordNET Editor \par\hspace{2em}{\small \url{http://wordventure.eti.pg.gda.pl/wne/wne.html}}
						\item WordVis \par\hspace{2em}{\small \url{http://wordvis.com/}}
						% \item GRAPH WORDS online thesaurus (\url{http://graphwords.com/})
					\end{itemize}

					% \begin{table}[]
					% 	\centering
					% 	\begin{tabular}{l}
					% 		An interactive visualization of the Princeton WordNet database                                                     \\
					% 		\hspace{2em}\url{http://mateogianolio.com/wordnet-visualization/}                                              \\\hline
					% 		Artha                                                                                                              \\
					% 		\hspace{2em}\url{http://artha.sourceforge.net/wiki/index.php/Home}                                             \\\hline
					% 		BabelNet                                                                                                           \\
					% 		\hspace{2em}\url{http://babelnet.org}                                                                          \\\hline
					% 		Cornetto Demo                                                                                                      \\
					% 		\hspace{2em}\url{http://cornetto.clarin.inl.nl/wordnet.xql?ssID=\&amp;word\_form=\&amp;pos=\&amp;sense\_pos=1} \\\hline
					% 		GoldenDict                                                                                                         \\
					% 		\hspace{2em}\url{http://goldendict.org/}                                                                       \\\hline
					% 		sloWTool                                                                                                           \\
					% 		\hspace{2em}\url{http://nl.ijs.si/slowtool/}                                                                   \\\hline
					% 		Treebolic                                                                                                          \\
					% 		\hspace{2em}\url{http://treebolic.sourceforge.net/}                                                            \\\hline
					% 		Visual Browser                                                                                                     \\
					% 		\hspace{2em}\url{https://nlp.fi.muni.cz/projekty/visualbrowser/}                                               \\\hline
					% 		wnbroswer\footnote{sic erat scriptum}                                                                            \\
					% 		\hspace{2em}\url{http://homepages.inf.ed.ac.uk/adubey/software/wnbrowser/index.html}                           \\\hline
					% 		WordNET Editor                                                                                                     \\
					% 		\hspace{2em}\url{http://wordventure.eti.pg.gda.pl/wne/wne.html}                                                \\\hline
					% 		WordVis                                                                                                            \\
					% 		\hspace{2em}\url{http://wordvis.com/}                                                                         
					% 	\end{tabular}
					% 	\caption{My caption}
					% 	\label{tab:my-label}
					% \end{table}

				Pro účely porovnání (určení východiska, baseline) bude do popisu zahrnuto ještě jedno rozhraní, a to oficiální vyhledávací rozhraní princetonského WordNetu (\url{http://wordnetweb.princeton.edu/perl/webwn}). 

			\section{Strukturalizace přehledu a kritéria hodnocení}

				Výše vypsaná rozhraní budou v dalších kapitolách rozdělena podle toho, zda umožňují přístup z webového prohlžeče bez nutnosti instalace doplňků (například pro běh zásuvných modulů napsaných v Javě\footnote{\url{https://docs.oracle.com/javase/tutorial/deployment/applet/}}), budou zhodnocena jejich positiva a negativa z hlediska použitelnosti pro získání informací o hledaném výrazu, komfort zacházení s rozhraním a případná omezení použitelnosti na různých zařízeních. 

				K rozdělení podle dostupnosti bylo přistoupeno jednak proto, že v době vzniku této práce je pokročilost webových technologií dostatečná k tomu, aby podobné vizualizace byly tvořeny jako webové stránky, a jednak proto, že cílem této práce je vytvořit všeobecně dostupné a použitelné rozhraní k wordnetům. Všeobecně použitelným je míněna použitelnost nejen na osobních počítačích, ale také na mobilních zařízeních, což v podstatě vyřazuje použití technologií jako jsou zásuvné moduly napsané v Javě či využívající Flash\footnote{\url{http://www.adobe.com/products/flashruntimes.html}} používané k realizaci různých existujících rozhraní (příkladem budiž Visual Editor využívající Javu). 

				Rozhraní jsou hodnocena na základě několika kritérií. Cílem je porovnat jejich přínos v kontextu ostatních existujících rozhraní a v kontextu současných trendů webových aplikací a ilustrovat, s jakými problémy se všeobecně rozhraní potýkají.  Z tohoto důvodu právě nebyla do přehledu zařazována rozhraní, která se podobají svou funkcionalitou a ovládáním rozhraním již zařazeným. Kritéria hodnocení v této práci jsou následující:

					\begin{itemize}
						\item přínos oproti základnímu oficiálnímu rozhraní princetonského WordNetu
							\begin{itemize}
								\item originální vizualizace dat vedoucí k možnosti identifikovat trendy v datech, které zůstávají uživateli skryty případě základní textové reprezentace dat
								\item reprezentace dat vhodnější z hlediska zásad přístupnosti webu\footnote{web accessibility}
							\end{itemize}
						\item responsivita rozhraní -- kvalitní použitelnost rozhraní i na zařízeních s menší obrazovkou, jako jsou mobilní zařízení typu chytrý telefon či tablet
						\item v případě webových aplikací ukládání stavu aplikace v adresním řádku prohlížeče (umožňuje sdílení nebo založení odkazu ke konkrétnímu hledání a zachování nastavení aplikace)
					\end{itemize}

			\section{Podmínky testování}

				Pro testování rozhraní, které bylo pro účely této práce provedeno, byl použit stolní počítač s monitorem disponujícím rozlišením 1920 × 1080 px, operační systém Ubuntu 16.04 LTS v 64bitové verzi, webový prohlížeč Pale Moon ve verzi 27.1.1 (rovněž 64bitový) s nastavaným uživatelským agentem na Firefox (z důvodů kompatibility). Ačkoliv se počítalo, vzhledem k nepříliš rozšířenému vykresovacímu jádru použitého prohlížeče Pale Moon, s možnými problémy v zobrazení některých rozhraní, nebyly tyto zjištěny, a tak nebylo nutné provádět testování rozhraní ještě v dalších prohlížečích. Lze předpokládat, že v rozšířených prohlížečích, jako jsou Firefox či Chrome, budou jednotlivá rozhraní fungovat a vypadat podobně jako v prohlížeči Pale Moon.

				Testování na mobilním chytrém telefonu bylo prováděno na zařízení Nexus 5 (displej s úhlopříčkou 5 palců a rozlišením 1080 × 1920 px) s operačním systémem Android 6 v prohlížeči Chrome (sestavení 57.0). Možnost \textit{request desktop site} byla nastavena na \textit{vypnuto}.

			\section{WordNet Search jako základ porovnání}
			\label{wnvis:wnsearch}

				Oficiální rozhraní k princetonskému WordNetu je koncipováno jako textová reprezentace dat, přičemž, byť volitelně, umožňuje zobrazovat veškeré k hledanému výrazu relevantní informace, jež WordNet obsahuje. Data jsou vizualizována jako soubor seznamů.

				Po vyhledání zadaného výrazu jsou uživateli prezentovány v bodech jednotlivé synsety obsahující hledanou slovní formu rozdělené podle slovních druhů, k nimž přináležejí. V základním nastavení každý bod sestává ze slovních forem tvořících daný synset, jeho definice, případnýho příkladu užití a odkazu na detaily synsetu. Kliknutím na tento odkaz se zobrazí v podseznamu sémantické a lexikální relace, v nichž je daný synset přítomen. Samotné relace jsou opět odkazy, jejichž otevřením je uživateli prezentován seznam synsetů, které jsou spolu s daným synsetem v dané relaci přítomny. Postupným otevíráním detailů synsetů v podseznamech se lze zanořovat hlouběji a hlouběji do struktury, resp. se ve struktuře vzdalovat původnímu synsetu po hranách relací.

				Uživatel má možnost si zvolit množství informací, jež jsou mu prezentovány, a to od úplného minima (pouze slovní formy nalezených synsetů a jejich syntaktické kategorie) až po vše, co se ve WordNetu vyskytuje, včetně identifikátorů a dalších technických detailů. Rozhraní svůj stav ukládá jako parametry URL\footnote{unikátní adresa stránky (\textit{Uniform Resource Locator}), obvykle se zobrazuje v adresním řádku prohlížeče}, takže je možné jej později obnovit z adresy.

				Rozhraní není responsivní, ale vzhledem k tomu, že je v základu relativně úzké (pod 700 px), je na chytrém telefonu použitelné.

				\begin{figure}[h]
					\centering
					\includegraphics[width=1.0\textwidth]{wnsearch.png}
					\caption{Ukázka rozhraní: WordNet Search 3.1 v základním nastavení a s některými otevřenými detaily u synsetu}
					\label{fig:wnsearch}
				\end{figure}

				Podobnou funkcionalitu a způsob zobrazení dat jako WordNet Search mají i některá další rozhraní. Za zmínku stojí například WNSearch\footnote{\url{http://www.golovchenko.org/cgi-bin/wnsearch}}, což je minimalistické čistě textové rozhraní, které díky absenci jakýchkoliv stylů je responsivní a dobře použitelné na mobilních zařízeních s libovolně velkým displejem. Mezi další rozhraní s podobnou funkcionalitou patří URCS Wordnet Browser (spustitelné s Javou)\footnote{\url{http://www.cs.rochester.edu/research/cisd/wordnet/}} či GoldenDict (rozhraní pro klasické počítače především s operačním systémem Windows)\footnote{\url{http://goldendict.org/}}.

		\chapter{Vizualizace s webovým rozhraním}

			Jako bylo naznačeno v úvodu této části, vizualizací s webovým rozhraním se v této práci míní taková implementace, která ze strany uživatele nevyžaduje instalaci žádných doplňujících aplikací či aplikačních prostředí (např. Javy) kromě samotného webového prohlížeče.

			\section{An interactive visualization of the Princeton WordNet database}
			\label{wnvis:intviswn}

				Jeden z projektů programátora jménem Mateo Gianolio z Lundské univerzity (Švédsko)\footnote{\url{https://www.linkedin.com/in/mateo-gianolio-a89558b6/?trk=profile-badge-name}}\td{je v poradku to linkovat takhle jako link na linkedin? nebo to mam napsat jako zdroj do citaci?}. Jde o jednoduché rozhraní napojené na princetonský WordNet, které po zadání hledaného výrazu zobrazí synsety obsahující daný výraz. Jednotlivé synsety jsou barevně odlišeny podle syntaktických kategorií (slovních druhů), k nimž náležejí, a uspořádány kruhově kolem hledaného výrazu. Z každého synsetu je vždy zobrazena jen první slovní forma. Pokud uživatel najede kursorem myši na některý ze synsetů, zobrazí se další případné slovní formy náležející do daného synsetu a jeho glosa (definice), je-li dostupná.

				Další slovní formy v nalezených synsetech jsou klikatelné, což umožňuje dostat se přes ně na synsety obsahující onu slovní formu, na níž bylo uživatelem kliknuto (s identickým výsledkem, jako kdyby dané slovo uživatel zadal do vyhledávacího pole).

				Ačkoliv úvodní odstavec na stránce s rozhraním vybízí uživatele k \uv{prozkoumání desambiguace slov} \parencite{GianolioWN}, zobrazují výsledky hledání pouze synsety a slovní formy do nich náležející, přičemž synsety připojené sémantickými vztahy hyponymie, meronymie etc. nejsou zobrazitelné (jinak než případným náhodným výskytem jedné slovní formy ve více synsetech). To použití WordNetu omezuje na obyčejný thesaurus.

				Samotný design grafického rozhraní je řešen poněkud nešťastně a působí dojmem, že cílem autora bylo prokázat své schopnosti používat různé standardní vizuální transformační funkce. Pro indikaci běžícího procesu vyhledávání byl použit efekt animace rotujícího vyhledávacího pole a po dokončení vyhledávání jsou výsledky zobrazeny paprskovitě okolo vyhledávacího pole. To způsobuje, že některé texty jsou zobrazeny pod úhlem až 90 stupňů, což může pro některé uživatele činit jejich přečtení obtížnějším. 

				Rozhraní není tzv. responsivní, tedy nepřizpůsobuje se velikosti obrazovky, na níž je zobrazeno. To komplikuje jeho použití na zařízeních s malou obrazovkou, kupříkladu mobilním telefonu.

				Toto rozhraní tedy poskytuje velice omezený přístup k datům WordNetu a neposkytuje žádné výhody oproti základnímu oficiálnímu rozhraní k princetonskému WordNetu. Jeho grafické pojetí je čistě arbitrární, neslouží žádnému účelu a naopak zhoršuje jeho použitelnost.

				\begin{figure}[h]
					\centering
					\includegraphics[width=0.8\textwidth]{intviswn.png}
					\caption{Ukázka rozhraní An interactive visualization of the Princeton WordNet database}
					\label{fig:intviswn}
				\end{figure}

			\section{WordNET Editor}
			\label{wnvis:wncoledit}

				Webová aplikace WordNET Editor byla vytvořena s cílem poskytnout internetové komunitě praktický a účinný nástroj pro rozvoj WordNetu. Její autoři, \textcite{szymanski2007cooperative}, argumentují, že vývoj WordNetu vyžaduje velké množství lidské práce, a je proto důležité jej tvořit tak, jako je tvořena například internetová encyklopedie Wikipedie, což znamená spoluprací nezávislých uživatelů. K tomu je však potřeba nástroj, který takovou spolupráci umožní, což podle nich oficiální nástroje princetonského WordNetu nejsou (mimo jiné pravděpodobně proto, že WordNet byl vyvíjen uzavřenou skupinou lidí \parencite{fellbaum2005wordnets}). Vedle editoru vztahů a synsetů však obsahuje WordNET Editor i prohlížeč wordnetu, pro který bylo rozhraní do tohoto přehledu zařazeno. Data, jež aplikace používá, jsou verzí princetonského WordNetu odvětvenou v jeho verzi 2.1. \parencite{szymanski2007cooperative}

				Rozhraní je rozděleno do levého sloupce a pravého (většího) \uv{plátna}, jež slouží pro grafické vyobrazení vztahů. Po vyhledání zadaného výrazu nabídne aplikace uživateli seznam synsetů, v nichž se výraz nachází. Nabídka synsetů je klikatelná a uživateli je po zvolení synsetu kliknutím prezentováno grafické vyobrazení daného synsetu se všemi jeho vztahy. V levém sloupci si pak uživatel může vybírat, další synsety, které chce do pravé sekce přidat, může také otevírat dvojklikem již zobrazené synsety a lze i v nastavení filtrovat, které vztahy a slovní druhy se mají zobrazovat. Aplikace však zřejmě neobsahuje intuitivní způsob jak zobrazené synsety z plátna vpravo odebírat a při otevírání dalších synsetů plátno nepřepisuje, ale synsety přidává k již zobrazeným (neprovázaně). To může vést ke značné nepřehlednosti reprezentace, avšak aplikace umožňuje přibližování a oddalování plátna a vizuální přesouvání zobrazení po plátně, což s jistým úsilím uživatele nepřehlednost může eliminovat. Také nutno vytknout absenci textového pojmenování vztahů mezi synsety. Barevné rozlišení je v tomto případě nevhodně zvolené, jelikož vztahů je v rozhraní hodně a některé barvy jsou obtížně odlišitelné.

				Grafické zobrazení vztahů mezi koncepty je řešeno hvězdicovitě, přičemž ve středu je zvolený synset, jenž byl výsledkem hledání, nebo na něj bylo kliknuto, a od něj jsou vedeny hrany představující vztahy k příbuzným synsetům. Zajímavý je originální koncept zobrazení synsetů příbuzných k hledanému (otevřenému) synsetu. Ten je v grafu reprezentován svými členskými slovními formami, ale příbuzné koncepty jsou reprezentovány svými definicemi. Trochu na závadu však je, že se zřejmě nedá (minimálně v prohlížecím režimu) zobrazit detail příbuzného synsetu a uživatel se dozví pouze jeho definici, nikoliv slovní formy do něj náležející. Jediný detail, který je uživateli dostupný, je u některých synsetů obrázek (o jeho zdroji se však v rozhraní nepíše).

				\begin{figure}[h]
					\centering
					\includegraphics[width=1.0\textwidth]{wneditor.png}
					\caption{Ukázka rozhraní WordNET Editor: zvolený synset ve středu grafu a zobrazený obrázek k příbuznému synsetu}
					\label{fig:wneditor}
				\end{figure}

				Test chování tohoto rozhraní na mobilním zařízení nebyl proveden, jelikož je zjevné, že je určeno především k editacím a podle toho je také navrženo. Rozhraní poskytuje grafický náhled na strukturu dat, ale pro nevhodné kódování vztahů do barev a nezobrazovaní dostatečného množství informací je jeho ovládání neintuitivní. Nutno podotknout, že rozhraní tak, jak je vyobrazeno v \textcite{szymanski2007cooperative}, vypadá značně odlišně a například obsahuje pojmenované vztahy.

				Podle webové stránky projektu\footnote{\url{http://wordventure.eti.pg.gda.pl}} je rozhraní vyvinuto s otevřeným zdrojem, pro jeho obdržení je však nutno kontaktovat vývojáře. Lze ale předpokládat jeho použitelnost i pro ostatní wordnety.

			\section{Cornetto Demo}
			\label{wnvis:cornetto}

				Cornetto Demo je prohlížeč pro nizozemský wordnet a jako jedno z mála dostupných webových aplikací kombinuje grafickou a textovou vizualizaci dat. To je dáno zřejmě mimo jiné tím, že Cornetto Demo není pouze rozhraním pro wordnet, ale kombinuje data ze dvou zdrojů. Jednak z Referentie Bestand Nederlands \parencite{martin2005referentie}, a jednak z nizozemského wordnetu. Referentie Bestand Nederlands je slovník obsahující informace podobně strukturované jako FrameNet \parencite{fillmore2004framenet} spolu s rozšířením o kombinatorickém chování slov v určitém významu \parencite{horak2008development}. 

				Rozhraní je rozděleno na tři moduly, základní vyhledávání, pokročilé vyhledávání a vizualizace synsetů. 

				Základní vyhledávání slouží k prostému vyhledávání lexikálních jednotek. Dotaz je možné základním způsobem omezovat či rozšiřovat, a to pomocí zástupných znaků (\ex{*}, \ex{?}\footnote{přičemž \ex{*} zastupuje posloupnost kterýchkoliv znaků, \ex{?} zastupuje jeden kterýkoliv znak}) a volby slovních druhů, v jakých se má výhledávat. \parencite{cornettoGettingStarted}

				Pokročilé vyhledávání umožňuje v nizozemském wordnetu najít lexikální jednotky, které mají společné specifické parametry. Jako kritéria pro vyhledávání lze zvolit kterékoliv kombinace všech příznaků, jež jsou ve wordnetu u lexikálních jednotek přítomny. Takto je možné například vyhledat všechna slovesa, která jsou řazena do domény tance a jsou označena jako archaická.  

				Pokud hledaný výraz či zvolená kritéria odpovídají některým slovům vyskytujícím se v databázi, je uživateli prezentován seznam nalezených jednotek, přes nějž se kliknutím na konkrétní jednotku lze dostat na její detail. Tyto podrobnosti jsou textovou reprezentací dostupných dat, tedy syntaktických a sémantických informací o vyhledaném slově (nutno podotknout, že značná část informací v této části zřejmě pochází spíše Referentie Bestand Nederlands, nikoliv z wordnetu) a hierarchického zařazení konceptu, do nějž slovo patří, ve wordnetu. Hierarchické zařazení je vyvedeno ve formě podobné tradičnímu znázornění adresářového stromu a obsahuje, zřejmě z úsporných důvodů, pouze přímé nadřazené a přímé podřazené koncepty a v rozhraní chybí úplné zobrazení cesty od kořene wordnetu k zobrazenému synsetu. I tak může strom být relativné dlouhý vzhledem k tomu, že u obecných hyperonym bývá seznam hyponym rozsáhlý. 

				\begin{figure}[h]
					\centering
					\includegraphics[width=1.0\textwidth]{cornetto-text1.png}
					\caption{Ukázka rozhraní Cornetto Demo: textová reprezentace dat}
					\label{fig:cornetto-text1}
				\end{figure}

				Z odkazů v hierarchii lze kliknutím na ikonu wordnetu u synsetu přejít do vizualizačního režimu, čímž se zobrazí grafické znázornění daného synsetu. 

				Modul vizualizace synsetů je v praxi pouze zobrazovací nadstavbou zbytku rozhraní, jelikož jeho vyhledávání se od základního vyhledávání liší pouze tím, že vybírá jeden nalezený výsledek a rovnou jej zobrazí graficky. Který výsledek se má vybrat, je možné zvolit při inicializaci hledání. Je tak eliminována nutnost zvolit synset, otevřít jeho detaily a kliknout na výše zmíněnou ikonu, aby se uživatel dostal na grafické znázornění daného synset, jedná se však pouze o zkratku v rozhraní, jež nezavádání nic nového.

				Samotná vizualizace je realizovaná pomocí javascriptové knihovny D3.js\footnote{\url{https://d3js.org}} \parencite{BostockD3} \td{nejsem si jist tou citaci - jak jsem prisel zrovna na tenhle dokument a zda neni lepsi dat jen link na D3.js do footnote} a umožňuje uživateli prohlížet všechny synsety svázané některým vztahem s vyhledaným synsetem. Zobrazení rozlišuje syntaktické kategorie i sémantické vztahy barevně, přičemž vztahy jsou navíc ještě popsány slovně. Autoři použili rozšířený model vizualizace synsetů tak, že synset je představován uzlem, k němuž vede sémantický vztah, a z tohoto uzlu pak vedou nepojmenované hrany představující vazbu slovní formy na daný synset (tedy v zásadě vztah synonymie). Tento způsob je alternativní k zobrazení, v němž sémantické vztahy vedou k jednotlivým slovním formám\td{asi prepsat, tohle nejspis (uz?) neni pravda, viz obrazek dole, plus pripsat pozn. o popiskach} a není tak na první pohled zřejmé, které navázané formy jsou součástí kterých synsetů. Zde použité zobrazení vhodně omezuje množství hran, které jsou nutné k vykreslení grafu, byť možná na úkor srozumitelnosti pro neznalého uživatele, jelikož body označující synset a hrany označující synonymii nejsou nijak označeny v grafu (pouze v legendě a při přejetí myší). Všechny uzly mají přiřazenou i vlastní legendu ve formě informační bubliny, která obsahuje například definici, příklady, slovní druh či identifikátory. Vizualizaci je možné kolečkem myši přibližovat a oddalovat a tažením myší se po zvětšeném grafu přesouvat. Funkčnost přibližování je možná poněkud diskutabilní, jelikož se při jejím použití nemění proporce zobrazovaných informací (cf. elektronické mapy na WWW, které při přiblížení sice zvětšují detail struktur, ale zachovávají velikost písmen v nápisech). Její smysl by v opačném případě tkvěl například v použití při vyhledání slov jako je \ex{plant}\footnote{niz. \ex{rostlina}}, které mají velké množství hyponym, a tudíž jejich graf je nepřehledný až do rozměrů naprosté nepoužitelnosti (obrázek \ref{fig:wncorplant} na straně \pageref{fig:wncorplant}). Ovládání klávesnicí možné není.

				\begin{figure}[h]
					\centering
					\includegraphics[width=1.0\textwidth]{wncorplant.png}
					\caption{Ukázka rozhraní Cornetto Demo: ilustrace přeplnění grafické reprezentace hyponymy u slova \ex{plant}}
					\label{fig:wncorplant}
				\end{figure}

				Rozhraní je neresponsivní, takže je na mobilním zařízení poněkud nepohodlné jej používat, byť to není nemožné. Šířka bloku s informacemi je natolik velká, že je nutné horizontálního rolování při čtení textu, což se považuje z hlediska použitelnosti webové stránky za velmi negativní \parencite{nn2005scrollbar, richards2004web}. Reprezentace grafem je vzhledem k ovládání, které používá, ještě hůře použitelná. Dotyk prstem do plochy s vykresleným grafem je totiž zároveň interpretován jako rolování celé stránky a zároveň jako tažení plátna s grafem do stran, které je běžně prováděno tažením myší po plátně. Přibližování a oddalování grafu, nikoliv celé stránky, v testovacím prostředí nebylo možné vůbec, ale vzhledem k vlastnostem této funkce popsaným výše to použitelnost nijak neomezuje. Co v mobilním testovacím prostředí v podstatě nefungovalo, byly informační bubliny u slov, jež se normálně zobrazují nad slovy; zde se zobrazovaly zcela mimo inkriminované slovo a částečně mimo zobrazovanou plochu.

				Oproti základnímu rozhraní WordNet Search 3.1 přináší toto rozhraní přehlednější textovou reprezentaci dat spojenou s vizualizací vztahů vyhledaného synsetu. Textové rozhraní však zobrazuje poněkud omezené množství informací z wordnetu a soustřeďuje se především na data ze slovníku Referentie Bestand Nederlands. Vztahy mezi slovy jsou zobrazovány pouze v grafické reprezentaci dat, což vzhledem k její nepříliš vysoké použitelnosti na mobilních zařízeních s menší obrazovkou může být omezující. Rozhraní uchovává svůj stav podrobně v URL a umožňuje jej z ní plně obnovit.


			\section{WordVis}
			\label{vis:wordvis}

				WordVis je rozhraní zaměřené na vizualizaci dat princetonského WordNetu grafem. Sestává z vyhledávacího pole, levého sloupce s výběrem synsetů a z plátna, na němž je vykreslen graf vztahů zvoleného synsetu nebo slovní formy. 

				Ačkoliv to na první pohled uživateli nemusí být zřejmé, vizualizace funguje ve dvou režimech. Prvním je zobrazení slovní formy a synsetů, v nichž se tato slovní forma vyskytuje, druhé je zobrazení synsetu jako centrální jednotky a vztahů a slovních forem, které k danému synsetu patří. První zobrazení je užíváno například při prvotním zobrazení výsledků hledání. Vizualizace zobrazí ve středu grafu vyhledanou slovní formu a připojí k ní nepojmenovanými hranami synsety, jež danou slovní formu obsahují. Pokud pak uživatel zvolí kliknutím myší některý konkrétní synset, zobrazí se ve středu vizualizace jeho značka, k níž jsou připojeny pojmenovanými hranami představujícími sémantické vztahy další synsety. Stejně jako v prvním zobrazení, i v tomto se slovní formy náležející k určitému synsetu připojují nepojmenovanými hranami jako textové uzly. Pokud jedna slovní forma náleží k více synsetům (např. substantivum \ex{bicycle} a sloveso \ex{bicycle}), je v grafu její textový uzel pouze jednou a vedou od něj dvě nepojmenované hrany k náležitým synsetům. 

				Hrany mezi synsety jsou orientované (znázorněné jako šipky) a k jejich pojmenování zvolil autor sémantické významy daných vztahů, tedy např. hyperonymie je značena slovem \textit{is}\footnote{angl. je}; např. \ex{apple tree \textit{is} fruit tree}\footnote{angl. \ex{jabloň \textit{je} ovocný strom}}.

				Vyhledávací formulář umožňuje filtrování výsledků hledání podle slovních druhů a typů vztahu. Graf je navíc aktualizován v reálném čase podle toho, jak jsou podmínky filtrování uživatelem měněny. Nevýhodou však je absence funkcí \textit{vybrat vše} a \textit{nevybrat nic}, která je citelná zvláště v případě, že uživatel chce vybrat pouze jeden druh relace (kterých je mnoho, tudíž odebrání všech relací z výběru kromě jedné může být časově náročné). Hledání také nabízí funkci napovídání poté, co uživatel zadá několik znaků z počátku slova. To umožňuje vybrat hledané slovo z nabídky a ušetřit několik úhozů, což je zvláště užitečné u delších slov či pro uživatele, který nevyhledává ve svém rodném jazyce a není si jist pravopisem hledaného slova.

				\begin{figure}[h]
					\centering
					\includegraphics[width=1.0\textwidth]{wordvis.png}
					\caption{Ukázka rozhraní WordVis (po zvolení konkrétního vyhledaného synsetu)}
					\label{fig:wordvis}
				\end{figure}

				Technologicky je vizualizace řešena javascriptovou knihovnou, jež vykresluje graf na plátně v HTML5, což je metoda generování grafiky na webových stránkách \parencite{w3schools2017htmlcanvas}. Uspořádání prvků na plátně je řešeno modelováním fyzikálních vlastností našeho světa. Jednotlivé uzly na plátně mají zadefinováno, že se navzájem odpuzují (podobně jako elektrony), naopak vazby mezi nimi fungují jako pružiny, jež mají nějakou ideální délku, v níž se snaží setrvávat. Díky těmto dvěma soupeřícím silám jsou body na plátně rozmístěny tak, aby využívaly prostor co nejúčinněji. Hrany navíc mohou mít zadanou preferovanou orientaci, což umožňuje orientovat například hyperonymní synsety směrem nahoru, hyponymní směrem dolů, etc. \parencite{wordvis2010vercruysse} Vzhledem k tomu, že však zřejmě v prostředí neexistuje žádná penalizace za křížení hran, je relativně běžné, že se hrany překříží. To značně snižuje přehlednost výsledného grafu, vzhledem k tomu, že pak není jasné, zda jsou hrany překřízeny z nějakého hlubšího důvodu (například jedna slovní forma náležející k více synsetům), či nikoliv. 

				Knihovna vykreslující graf jej umožňuje uživateli modifikovat tažením jednotlivých uzlů myší, prodlužovat či zkracovat hrany a podobně. V aplikaci pro zobrazení dat z WordNetu toto pravděpodobně nehraje příliš zásadní roli a přítomnost této funkcionality je zřejmě dána tím, že knihovna ji zajišťující je určena pro co nejširší použití a snaží se být univerzální. \parencite{wordvis2010vercruysse}

				Rozhraní je neresponsivní a na mobilním zařízení je nutné k zajištění čitelnosti textů na stránce obsah přiblížit, což vede k nutnosti rolovaní jak vertikálnímu, tak horizontálnímu. To má opět dopad na použitelnost webové stránky minimálně na mobilních zařízeních s malou obrazovkou \parencite{nn2005scrollbar, richards2004web}. Také není možné využít funkcionality přesouvání prvků na plátně, což ale při použití na vizualizaci dat z wordnetu není velkým omezením funkčnosti. Co se udržování stavu aplikace v URL týče, je implementováno pouze čtení parametru {\tt q}\footnote{odvozeno od \textit{query}, angl. \textit{dotaz}}, zpětně do něj ale už zapisováno není. Odkazy na synsety na levé straně tento parametr obsahují, stejně tak jej lze zkopírovat přes kontextové menu pro každý uzel v rozhraní, takže teoreticky je možné obnovit kterýkoliv stav aplikace, byť dostupnost této funkcionality není příliš intuitivní.

				Přínos tohoto rozhraní tkví především v relativně přehledné vizualizaci dat princetonského WordNetu, nutno však podotknout, že pro některé své vlastnosti nemusí pro nezkušeného uživatele být zpočátku zcela jednoduše použitelné a dobře přehledné (např. pro zmíněné křížení hran).

				Na knihovně, na níž je toto rozhraní postaveno, stojí ještě některá další rozhraní k princetonskému WordNetu, například VisuWords\footnote{\url{http://www.visuwords.com/}, \url{https://www.linux.com/news/visuwords-wordnet-goes-graphical}} či Visual Thesaurus\footnote{\url{https://www.visualthesaurus.com}}. Ta nebyla do hodnocení v této práci zahrnuta, protože jsou funkčně podobná rozhraní WordVis.

			\section{sloWTool}
			\label{vis:slowtool}

				Rozhraní sloWTool bylo vyvinuto pro potřeby slovinského wordnetu, jenž je založen na princetonském WordNetu a vznikl skombinováním několika zdrojů, například Wikipedie, dvojjazyčných slovníků či paralelních korpusů. Je víceúčelovým nástrojem, který umožňuje textovou reprezentaci, vizualizaci a editaci dat z wordnetu. V době psaní této práce v něm bylo podle rozbalovací nabídky vedle vyhledávacího pole možno pracovat s anglickým, francouzským, polským a slovinským wordnetem, testováno bylo rozhraní s anglickým. Tvůrcům rozhraní sloužilo k revizím slovinského wordnetu po jeho rozšiřování automatickými nástroji. \parencite{fivser2012slownet} Cílem při vývoji tohoto rozhraní podle \textcite{fivser2011visualizing} bylo překonat nevýhody tehdejších dostupných rozhraní a vyvinout nástroj, který by mezi jiným umožňoval prohlížení i editaci, spolupráci mnoha autorů včetně anonymních editací (záměr tvůrců byl využít náhodných návštěvníků k opravování chyb, jež ve wordnetu naleznou) či možnost jednoduché registrace uživatelů. Mezi dalšími podmínkami byla možnost přidávání dalších wordnetů do systému a s tím spojená schopnost rozhraní zobrazovat vícejazyčná data. V neposlední řadě se autoři v kontextu tehdejších nástrojů také snažili zaměřit na platformní nezávislost a přenositelnost rozhraní. 

				Rozložení stránky je vzdáleně podobné tomu u základního oficiálního rozhraní k princetonskému WordNetu, a to v tom smyslu, že neobsahuje relativně rozšířený levý sloupec na výběr synsetů, ale jednotlivé významy, které jsou nalezeny po zadaní hledané slovní formy do vyhledávacího pole, seskupuje do sekcí v hlavním bloku spolu s textovou reprezentací dat. Vyhledávací pole podporuje napovídání v databázi existujících slovních forem, podobně jako je to zařízeno v rozhraní WordVis, což může usnadnit práci s vyhledáváním. V textové reprezentaci rozhraní zobrazuje zřejmě všechny dostupné informace o synsetech, tedy se chová podobně jako základní rozhraní princetonského WordNetu, je-li nastaveno tak, aby nefiltrovalo žádné informce. U jednotlivých vztahů, v nichž je daný synset přítomen, pak je možné kliknutím na ikonu šipky otevřít synset nacházející se pomyslně na druhé straně daného vztahu (např. \ex{bicykle}\footnote{angl. \ex{cyklistické kolo}} má meronymum \ex{saddle}\footnote{angl. \ex{sedlo}}, takže lze otevřít detail synsetu \ex{saddle}). To trpí stejným neduhem, jako základní rozhraní princetonského WordNetu, to jest, že lze ve smyčce otevírat nadřazený a podřazený synset a stále se zanořovat do smyčky hlouběji. To je nejen nesmyslné, ale zároveň to v implementaci tohoto rozhraní postupně může začít zpomalovat rychlost reakcí prohlížeče pro nadměrné množství uzlů v DOM\footnote{document object model, model objektů v dokumentu}.

				Po levé straně se nachází lišta s ikonami odkazujícími na dalšími moduly, které se otevírají v emulaci nových oken (v témže panelu webového prohlížeče). Mezi tyto moduly patří mimo jiné i pokročilé hledání, vizualizace dat a nápověda. 

				Pokročilé hledání funguje podobně jako u ostatních rozhraní, tedy umožňuje používat zástupné znaky (\ex{*} a \ex{?}), filtrovat slovní druhy, nebo vyhledávat v jiných polích než jsou slovní formy patřící do synsetů (například v definicích či podle identifikátoru synsetu). 

				Vizualizace, modul z hlediska této práce nejpodstatnější, je určen k zobrazení grafické interpretace vztahů vyhledanýho synsetů. Na rozdíl od ostatních rozhraní tato vizualizace zobrazuje všechny vyhledané synsety (s kořenem grafu označeným červeně a s textem hledaného výrazu) a neumožňuje jejich další filtrování v zobrazení. Také nepodporuje kromě přesouvání prvků na plátně a zvýraznění příslušného synsetu v hlavním bloku textové reprezentace poté, co je na něj ve vizualizaci kliknuto, zřejmě žádnou formu interakce. Pro srovnání, u ostatních vizualizací lze narazit například na informační bubliny, jež zobrazí podrobnosti jednoho prvku. Pokud uživatel potřebuje zobrazit vizualizaci jednoho konkrétního synsetu, je nucen použít pokročilé vyhledávání a vyhledat daný synset nejlépe podle jeho identifikátoru, který je vždy unikátní. Vizualizace je zatížena také dalšími problémy, jako jsou nedostatečné možnosti přizpůsobování velikosti zobrazovací plochy (zvětšování okna je podporováno, ale plátno s grafem zůstává konstantně velké), občasnou desynchronizací obsahu vizualizace s výsledky hledání, náročností na uživatelovo technické vybavení počítače (především výpočetní jednotku) a značnou nepřehledností způsobenou nedostatečně či nevhodně řešenou penalizací překrývání prvků na zobrazovací ploše. 

				\begin{figure}[h]
					\centering
					\includegraphics[width=1.0\textwidth]{slowtool.png}
					\caption{Ukázka rozhraní sloWTool (s otevřenou grafickou vizualizací)}
					\label{fig:slowtool}
				\end{figure}

				Rozhraní je svými možnostmi zprostředkování informací z wordnetu poměrně inovativní, reálná použitelnost ovšem trpí zmíněnými nedostatky a odchyluje jej tím od původního záměru autorů, který minimálně zčásti zůstává nenaplněn. Je sice pravdou, že \textit{s aplikací je možné pracovat v každém moderním přihlížeči, ať už na počítači, tabletu, či dokonce mobilním telefonu} \parencite{fivser2011visualizing}, nutno ale podotknout, že rozhraní je zcela neresponsivní a jeho ovládací prvky jsou zcela nevhodné pro ovládání na menší obrazovce a dotykem.

				Stav aplikace lze do \td{prepsat cast o obnovovani obsahu, je to uz cele trochu slozitejsi}jisté míry uložit pomocí odkazu \textit{Link}, který vede na adresu, z níž lze obnovit hledané slovo (konfiguraci otevřených nástrojů však už nikoliv).

				Navzdory všem nedostatkům této aplikace je dlužno uznat, že se svou univerzalitou a funkcionalitou značně přibližuje rozhraní, které je praktickým cílem i této práce.

				SloWtool bylo vytvořeno pod licencí Creative Commons\footnote{\url{https://creativecommons.org/licenses/by-nc-sa/3.0/}} a jeho zdrojový kód je dostupný na platformě pro aplikace s otevřeným zdrojem Launchpad\footnote{\url{https://launchpad.net/slowtool}}.

			\section{BabelNetXplorer}
			\label{vis:babel}

				BabelNet je rozsáhlý projekt vícejazykové sémantické sítě, který čerpá data z více zdrojů. Záměrem autorů bylo při tvorbě této sémantické sítě 
				% poskytnout vědecké komunitě i široké veřejnosti volně dostupný multilinguální zdroj provázaných informací a soubor nástrojů, které umožní s těmito informacemi pracovat. 
				eliminovat základní faktory definující nevýhody tehdejších (a potažmo i aktuálních) mezinárodních projektů zabývajících se sémantickými sítěmi. Těmi jsou manuální tvorba dat, které sítě konstituují, a s tím související nerovnoměrnost množství dat přes jednotlivé jazyky. Jazyky s vysokou hustotou zdrojů, jakým je například angličtina, tak ve výsledku mají více dat i v sémantické síti. Autoři BabelNetu se pokusili tento problém překonat kombinací několika metod, jejichž společným prvkem je automatizace. Informace o významech jsou v BabelNetu doplňovány z Wikipedie, která je podle autorů díky mnohačetným zásahům expertů z různých oborů ve výsledku přesným a informačně bohatým zdrojem. Druhá důležitá metoda, zaměřující se především na nerovnoměrnost dat v různých jazycích, je automatický překlad zdrojů. \parencite{navigli2010babelnet}

				BabelNetXplorer je webové grafické rozhraní vytvořené pro vizualizaci dat z BabelNetu. \textcite{navigli2012babelnetxplorer} uvádějí, že rozhraní slouží k vizualizaci vztahů pro slova nalezená v BabelNetu a ilustrují vzhled rozhraní dvěma snímky obrazovky. V době vzniku tohoto textu však rozhraní BabelNetXploreru už vypadá výrazně odlišně, což je vzhledem k tomu, že od vzniku práce \textcite{navigli2012babelnetxplorer} uběhlo pět let, pochopitelné. V této práci bude z evidentních důvodů rozebrána použitelnost a funkcionalita současné verze.

				Rozhraní obsahuje klasicky vyhledávací pole, vedle něhož se nacházejí rozbalovací nabídky, v nichž si uživatel může vybrat jazyk, v němž chce vyhledávat, a do kterého jazyke chce slovo případně přeložit. Po úspěšném dokončení vyhledávání jsou uživateli zobrazeny odkazy na jednotlivé synsety v seznamu, který není nepodobný ostatním rozhraním. Podstatným rozdílem však je, že se v něm zobrazují i ilustrační obrázky, které, byť nejsouce vždy velmi informativní, mohou napomoci uživateli v orientaci, který synset jej zajímá. 

				V detailu synsetu, který se zobrazí po kliknutí na příslušný odkaz (kterým je vždy nadpis každé položky zmíněného seznamu), je uživateli zobrazen seznam slovních forem, které k danému synsetu náleží, jeho definice (s možností zobrazit i jeho definice nejen z daného wordnetu, v němž uživatel vyhledává, ale i z Wikipedie a dalších zdrojů), jeho případný překlad a sémantické vztahy, do nichž tento synset náleží. K disposici má uživatel i možnost zobrazit si daný synset paralelně v dalších jazycích pomocí nabídky pod vyhledávacím polem (viditelné na snímku obrazovky \ref{fig:babelxplorer} na straně \pageref{fig:babelxplorer}). Níže na stránce jsou uživateli prezentovány informace z dalších zdrojů napojených na BabelNet, jako jsou obrázky, překlady, odkazy na Wikipedii, etc. a odkaz na vizuální reprezentaci sémantických vztahů otevřeného synsetu. 

				\begin{figure}[h]
					\centering
					\includegraphics[width=1.0\textwidth]{babelxplorer.png}
					\caption{Ukázka rozhraní BabelXplorer (detail synsetu v textové reprezentaci)}
					\label{fig:babelxplorer}
				\end{figure}

				Vizuální reprezentace je řešena tradičním hvězdicovitým grafem, který má dva režimy. Jeden zobrazuje vedle textových názvů jednotlivých synsetů přítomných v grafu i jejich zástupné obrázky, druhý je čistě textový a, nutno podotknout, značně přehlednější. Uzly reprezentující synsety jsou provázány barevně odlišenými hranami, kde barvy značí druh vztahu, který dva synsety spojuje. Uzly jsou klikatelné, přičemž po kliknutí na některý z příbuzných synsetů se zobrazí tento synset a opět hvězdicovitě další synsety, s nimiž je zkoumaný synset provázán vztahy. Při najetí korsorem myši na určitý synset se zobrazí jeho detail; tyto informační bubliny jsou však vázány na pozici kursoru myši a nelze tak kursorem najet do detailu tak, aby bylo možné kliknout na odkazy, které jsou v informační bublině obsaženy, a dostat se tak na textovou reprezentaci daného synsetu. Zdá se tedy, že není možné přejít z grafické reprezentace do textové.

				Poněkud nešťastně je řešen design vizualizace, jelikož jednak je možné graf přibližovat a oddalovat pouze kolečkem myší (které je na mobilních telefonech samozřejmě nedostupné), a jednak pokud uživatel nenajede na konkrétní synset kursorem myši, graf se samovolně neustále otáčí, což ztěžuje orientaci v něm.

				Rozhraní všeobecně responsivní, textová reprezentace je dobře použitelná i na mobilním rozhraní s malou obrazovkou. Grafická reprezentace je však omezena na větší obrazovky, jelikož v ní není možné graf posouvat po obrazovce, a tudíž jeho značná část (byť se částěčně přizpůsobuje zobrazovací ploše zřejmě v závislosti na velikosti obrazovky zařízení) zůstává uživateli skryta. Pro dotyková zařízení také není vhodné spoléhání na přejetí myší přes uzel pro zobrazení informační bubliny, jelikož to na dotykových obrazovkách také není příliš dobře proveditelné (kvůli neschopnosti rozlišit kliknutí a \uv{přejetí kursorem}).

				V době vzniku této práce mělo rozhraní dvě verze, z nichž jedna byla označana jako \textit{živá beta}, druhá jako \textit{současná (3.7)}, avšak při testování nebyly zjištěny zásadní rozdíly mezi těmito verzemi. Nutno ovšem podotknout, že v testovacím prohlížeči na klasickém počítači vykazovala verze beta minoritní chyby v rozložení stránky.

		\chapter{Vizulizace v aplikačním prostředí Java}

			Java je objektově orientovaný programovací jazyk, který je zaměřený na to, aby měl co nejméně závislostí na konkrétním operačním systému a technickém vybavení počítače, jak jen to je možné. To je výhodné zejména kvůli tzv. konceptu WORA (\textit{write once, run anywhere}, tedy \textit{napsat jednou, spustit kdekoliv}), protože to znamená, že skompilované programy napsané v Javě lze spustit na jakémkoliv stroji, který podporuje Javu. Takto skompilované programy se spouští v aplikačním prostředí zvaném Java Virtual Machine\footnote{angl. virtuální stroj Javy} (JVM), jež vytváří uniformní prostředí pro běh aplikací. Výhodou Javy tedy je velká přenositelnost programů v tomto jazyce napsaných, jelikož jednou skompilovaný kód lze spustit na různých operačních systémech a zařízeních. Nevýhodou je, že si uživatel, který chce program v Javě napsaný spustit, musí nainstalovat aplikační prostředí Javy (JVM). 

			\td{tenhle odstavec nemam ozdrojovan, je to pravda? viz zdroj za chvilku, neco tam o tom malo je}Java je velmi oblíbená pro vývoj aplikací, které fungují podle modelu klient--server, což je uspořádání, v němž data má uložená centrální server, k nemuž se připojují uživatelské klienty, odesílají mu své požadavky a uživatelům zobrazují vrácená data. \parencite[13]{gosling1995java} Na podobném principu může fungovat i rozhraní pro wordnet, pokud jsou data wordnetu uložena na serveru.

			Vizualizace vytvořené pro prostředí Java nemá příliš smysl hodnotit z hlediska responsivity, jelikož a priori nejsou vytvářena pro mobilní zařízení (až na výjimky, které nebylo možné zhodnotit z důvodu jejich nedostupnosti v době vzniku této práce). Co však bráno v potaz být může, je jejich schopnost přizpůsobit se velikosti obrazovky. Jakkoliv tento parametr může se zdát být samozřejmým, ukázalo se, že ne všechna rozhraní jsou na různé velikosti obrazovek osobních počítačů přizpůsobena.

			\section{wnbroswer}

				\begin{figure}[h]
					\centering
					\includegraphics[width=1.0\textwidth]{wnwordnetbrowswer.png}
					\caption{Ukázka rozhraní wnbroswer}
					\label{fig:wnwordnetbrowswer}
				\end{figure}

				Rozhraní wnbroswer\footnote{sic erat scriptum, název rozhraní je psán na různých místech různými způsoby, více v poznámce na konci této sekce a na obrázku \ref{fig:wnbroswerbadwriting} na straně \pageref{fig:wnbroswerbadwriting}} je určeno k prezentaci dat z princetonskému WordNetu, které pracuje pouze s grafovou grafickou reprezentací. Bylo vytvořeno pro reprezentaci dat z WordNetu ve verzi 3.0 a vyšší a vyžaduje lokální instalaci dat. Rozhraní je zřejmě svázáno se vznikem rumunského wordnetu \parencite{fivser2011visualizing}. Sestává z vyhledávacího pole, volby, mezi kterými syntaktickými kategoriemi se má vyhledávat, a selektoru na typy vazeb, jež mají být zobrazeny. Po úspěšném vyhledání konkrétního slova se zobrazí graf konceptuálně podobný grafu z rozhraní sloWTool, tedy ve středu je uzel reprezentující hledaný výraz, z něhož vedou hrany k jednotlivým synsetům, jež danou slovní formu obsahují. 

				Grafická reprezentace je interaktivní, lze na ní zvolit, jaký druh sémantického vztahu se má zobrazit, a lze tak u každého uzlu postupně zobrazit všechny synsety, které jsou s ním provázány; toho lze docílit buď přes kontextové menu, nebo přes volby vztahů v levém sloupci rozhraní, v němž se mimo jiné nachází i vyhledávací pole. Rozhraní zobrazuje i detaily zvoleného synsetu (po kliknutí na daný synset), takže až na technické informace typu číslo významu a podobně je schopno zobrazovat všechna dostupná data.

				Jakkoliv velikost okna s aplikace měnit lze, velikost plochy pro vykreslování grafu zůstává neměnné velikosti, lze tedy v terminologii webových aplikací říci, že je neresponsivní, jelikož se nedokáže přizpůsobit velikosti obrazovky zařízení uživatele.

				Aplikaci je možné používat bez připojení k Interetu, jelikož spoléhá na data WordNetu nainstalovaná lokálně. 

				\textbf{Poznámka:} Poněkud komicky působí fakt, že rozhraní má tři různé názvy: wnbroswer (hlavní nadpis stránky projektu), WordNet Browswer (titulek okna aplikace) a wnbrowser (ostatní výskyty). Ilustruje to obrázek \ref{fig:wnbroswerbadwriting} na straně \pageref{fig:wnbroswerbadwriting} pořízený 7. 8. 2017.

				\begin{figure}[h]
					\centering
					\includegraphics[width=0.7\textwidth]{wnbroswer.png}
					\caption{Ukázka rozdílného psaní názvu rozhraní na jeho webových stránkách}
					\label{fig:wnbroswerbadwriting}
				\end{figure}

			\section{Treebolic}

				\begin{figure}[ht]
					\centering
					\includegraphics[width=1.0\textwidth]{wntreebolic.png}
					\caption{Ukázka rozhraní Treebolic (se zobrazenou informační bublinou na synsetu)}
					\label{fig:wntreebolic}
				\end{figure}

				Treebolic je aplikace určená k zobrazování hierarchických dat v hyperobilické reprezentaci, která zajišťuje, že důležitá data v centru obrazovky jsou zobrazena uživateli ve větším měřítku, než data na okrajích grafu. Reprezentaci si lze představit jako plochu zobrazenou přes čočku. \parencite{boutreebolic}

				Zobrazení používá model prezentující vše, co je nalezeno k hledanému slovu, tedy při úspěšném vyhledání výsledků se uprostřed vizualizace zobrazí reprezentace vyhledávání, s níž jsou hranami provázáany všechny nalezené synsety. Oproti ostatním zde popisovaným prostředím navíc seskupuje tyto synsety podle slovního druhu, tedy zavádí další metakategorii v zobrazení, což přispívá k přehlednosti. 

				Co jí naopak ubírá, jsou poněkud kryptické ikony sémantických vztahů, jejichž význam je sice popsán na hraně, kde jsou dané ikony umístěny, popřípadě je možno jej odhalit pomocí nabídky \textit{Info} v kontextovém menu po kliknutí pravým tlačítkem danou ikonu, ale při prvním setkání se toto řešení nezdá být příliš ergonomickým (mimo jiné proto, že duplikuje informace). Ne zcela evidentní jsou také číselné a procentuální údaje v místě uzlů se synsety. Uživatel se sice díky řešení, které používá konexe hranami ze synsetů k členským slovním formám, dozví, o který synset jde (díky oněm členským slovním formám), ale význam zmíněných údajů zůstává skryt. Výhodou je barevné kódování významů jednotlivých hran. Barvy jsou navíc editovatelné v nastavení, takže teoreticky rozhraní vyhoví i osobám se zhoršeným vnímáním barev.

				Aplikace byla vyvinuta i pro Android (byť tato verze v rámci práce netestována nebyla), tudíž lze říci, že možná jistým zprostředkovaným způsobem splňuje požadavky na responsivitu.

				Aplikaci je možné používat bez připojení k Interetu, jelikož spoléhá na data WordNetu nainstalovaná lokálně.

				Teoretický přínos této implementace spočívá v pohledu na stratifikaci důležitosti vizualizovaných dat. Idea hyperbolického zobrazení rozvinutá dále především v oblasti ovladatelnosti by mohla být správným krokem prezentace většího množství dat, než je schopna obrazovka (a potažmo zorné pole) uživatele pojmout.

			\section{VisualBrowser}

				\td{napsat diplomaticteji, nebo se pokusit aplikaci rozjet poradne}Verze aplikace VisualBrowser testovaná v rámci této práce neprokázala dostatečnou intuitivitu a funkčnost, aby její vlastnosti mohly být dále rozebrány. 

		\chapter{Aplikace pro klasické počítače}

			Aplikace pro klasické počítače (jinak také desktopové aplikace) jsou takové aplikace, které jsou určeny pro běh nativně na operačním systému uživatelova počítače. Nativně spouštěná aplikace je, ať už přímo, či nepřímo, závislá na programových vlastnostech operačního systému daného stroje, stejně jako zprostředkovaně na jeho technickém vybavení. Přístupy k distribuci desktopových aplikací jsou v zásadě dva; první, běžný hlavně pro operační systém Windows, spočívá v distribuci již skompilovaných binárních souborů, které zajistí instalaci daného programu do systému a uživateli zbývá jen příslušný program spustit. Druhý způsob, běžný spíše pro různé linuxové distribuce, je založený na distribuci souborů obsahujících zdrojový kód programu, přičemž uživatel je nucen, má-li zájem program používat, si binární spustitelný soubor skompilovat. Druhý způsob pro programátora skýtá zásadní výhodu v tom, že se nemusí zajímat o prostředí, na němž bude koncový uživatel jeho program provozovat, protože kompilace programu pro všechny dostupné konfigurace (kombinace progamového a technickéh vybavení koncového uživatele) je náročná na zdroje. \parencite{Elizabeth2015} Pro onoho koncového uživatele však slyne druhá varianta tou nevýhodou, že kompilace programového vybavení vyžaduje z jeho strany netriviální usilí a znalosti o vlastním systému. 

			V rámci rozhraní určených pro zprostředkování dat z wordnetů jsou desktopové aplikace nejméně zastoupeny. Je tomu pravděpodobně proto, že jsou nejnáročnější na vývoj, zvláště mají-li být dostupné pro všechny majoritní desktopové operační systémy. Dalším důvodem může být fakt, že často bývá zvolen model klient--server, pro který jsou vhodnější spíše aplikace v Javě \td{jinde stranky neuvadim, co s tim? bez ni bych to tam nenasel, je to dlouhy}\parencite[13]{gosling1995java} či webová rozhraní.

			\section{Artha}

				\begin{figure}[h]
					\centering
					\includegraphics[width=1.0\textwidth]{wnartha-ubuntu.png}
					\caption{Ukázka rozhraní aplikace Artha}
					\label{fig:wnartha-ubuntu}
				\end{figure}

				Artha je multiplatformí \td{check kniz. prirucka, jestli tohle smim, dle internet. to neexistuje}thesaurus založený na WordNetu 3.0 \parencite{ramaswamy2012}. Grafické prostředí aplikace je vytvořeno v GTK\footnote{multiplatformní soubor nástrojů pro tvorbu grafických rozhraní (\url{https://www.gtk.org/})}, takže umožňuje (minimálně na testovací konfiguraci, tedy operační systém Ubuntu 16.04) komfortní práci. Je možno ji nainstalovat z repozitářů (tedy nevyžaduje náročnou kompilaci\footnote{podobně jako při distribuci skompilovaných binárních souborů}), což je pro koncového uživatele velmi pozitivní. Dle dokumentace je k disposici instalátor i na operační systém Windows\footnote{\url{https://www.microsoft.com/en-us/windows/}}. O Apple macOS\footnote{\url{https://www.apple.com/lae/macos}} se dokumentace nezmiňuje. Funguje bez připojení k Internetu a zobrazení je zaměřeno především na zobrazení významů synsetů (význam je reprezentován definicemi). Problémem je ukládání stavů aplikace, který se projevuje na nepříliš propracované funkcionalitě možnosti kroku zpět (na předchozí vyhledávání). Pokud například uživatel klikne dvojklikem na vybrané slovo v definici, Artha se jej pokusí\footnote{selže, pokud uživatel vybere před dvojklikem více slov} přesměrovat na definici tohoto slova ve WordNetu, ale při kroku zpět už mu zobrazí slovo, jež hledal \textit{před} slovem, z nějž se dostal na \textit{současnou} definici. Oproti základnímu rozhraní WordNetu nabízí Artha větší pohodlí v podobě desktopové offline aplikace, což pro uživatele znamená, že není při používání tohoto rozhraní závislý na připojení k Internetu.

				Artha zobrazuje data čistě textově (v tom smyslu, že nenabízí žádnou formu vizualizace), přičemž se rozhraní dělí na dvě části, a to nalezené synsety s blizšími informacemi a sémantické vztahy, které se k vybranému synsetu vážou (ilustrováno na snímku obrazovky \ref{fig:wnartha-ubuntu} na straně \pageref{fig:wnartha-ubuntu}). 

			\section{GoldenDict}
 		
 				\begin{figure}[h]
					\centering
					\includegraphics[width=1.0\textwidth]{wngoldendick-ubuntu.png}
					\caption{Ukázka rozhraní aplikace GoldenDict}
					\label{fig:wngoldendick-ubuntu}
				\end{figure}

				Podobně jako Artha funguje GoldenDict, který, jsa univerzálnějším rozhraním pro více lexikografických zdrojů, byl testován jen ve své verzi pro WordNet. GoldenDict zobrazuje čistě textovou reprezentaci umožňující kliknutím zobrazit detaily jednotlivých synsetů. Slovníková data si obstarává aplikace sama, takže uživatel nemusí podstupovat instalaci dat princetonského WordNetu (která může být netriviální, protože například pro Windows neposkytují oficiální webové stránky instalátor pro data WordNetu ve verzi 3). Zároveň však rozhraní umožňuje přidávat další slovníky a v základu umožňuje zobrazovat výsledky hledání z Wikipedie. 

				Rozhraní obsahuje levý sloupec pro zobrazování výsledků hledání a společně s hlavní plochou pro zobrazování detailů synsetu se chová podobně jako ostatní rozhraní (ilustrováno na snímku obrazovky \ref{fig:wngoldendick-ubuntu} na straně \pageref{fig:wngoldendick-ubuntu}). Aplikace udržuje historii hledání, umožňuje hledat slova z definic synsetů pomocí dvojkliku kursorem myši a dává uživateli k disposici relativně široké možnosti konfigurovatelnosti. Její rozhraní se dobře přizpůsobuje velikosti obrazovky (potažmo okna).

				Grafické rozhraní aplikace GoldenDict je postaveno na Qt\footnote{multiplatformní soubor nástrojů pro tvorbu grafických rozhraní (\url{https://www.qt.io/ui/})} a pro zobrazování dat z webu používá WebKit\footnote{vykreslovací jádro pro webové prohlížeče (\url{https://webkit.org/})}. \parencite{goldendict2016}\td{tady by to chtelo dopsat vic, aby to bylo konsistentni se zbytkem, nejaky zkusenosti z pouzivani}


		\chapter{Shrnutí přehledu rozhraní a vyplývající závěry o vhodném rozhraní k sémantickým sítím}
		\label{cha:shrnuti-prehledu}
		\chaptermark{Shrnutí přehledu}

			% z hlediska pouzitelnosti byva lepsi webovy rozhrani, bo to uzivatel nemusi instalovat (some fuken links for that)
			% java je sice kchul, ale porad ji musi mit uzivatele nainstalovanou, asi lidi moc nemaji (applety jsou uplne k nicemu uz, FF to prestal podporovat)
			% drtiva vetsina rozhrani neni responsivni, coz je problem, kdyz vetsina lidi pouziva web z mobilu: http://bgr.com/2016/11/02/internet-usage-desktop-vs-mobile/
			% je potreba rozhrani, ktery bude webovy, vypadat konsistentne na vsech zarizenich a bude splnovat, co se od nej ocekava (designove) - co se ocekava od WN browseru? dunno, but it's important - https://conversionxl.com/why-simple-websites-are-scientifically-better/

			Rozhraní v přehledu prezentovaném v této práci byla hodnocena dle různých aspektů odpovídajícím účelu a provedení dané třídy rozhraní. Nejpodrobněji byla hodnocena webová rozhraní. U těch byla jako největší společný nedostatek identifikována nedostatečná responsivita, což vede k tomu, že uživatel dané rozhraní nemůže (nebo může velmi omezeně) používat na mobilním zařízení. To v době, kdy počet uživatelů přistupujících k webovým službám přes mobilní zařízení je vyšší než počet těch, kteří k nim přistupují pomocí klasických počítačů \parencite{Heisler2016}, aplikaci značně diskriminuje. Z podobného důvodu jsou v nevýhodě také rozhraní vytvořená pro aplikační prostředí Java a v ještě větší míře aplikace vytvořené přímo pro operační systémy klasických počítačů. 

			Aplikace vytvořené v Javě jsou sice teoreticky spustitelné i na některých mobilních operačních systémech (zejména Android \parencite{SX92854}, u ostatních je podpora omezená či žádná (\textcite{SX15501535}, \textcite{SX1193541} inter alia), prakticky je jejich použitelnost vzhledem k jejich designu grafického rozhraní omezena ale také výhradně na klasické počítače. 

			Pro některé aplikace vyvinuté v Javě byly vytvořeny tzv. aplety Java\footnote{\url{https://nlp.fi.muni.cz/projekty/visualbrowser/applet/index.html}}, ale vzhledem k tomu, že některé silně rozšířené webové prohlížeče tento druh zásuvných modulů už vůbec nepodporují \parencite{MozzilaFoundation2017}, je jejich použití značně omezené a do budoucnosti lze předpokládat plošnou nepoužitelnost. 

			Rozhraní závislá na platformě (operačním systému, případně aplikačním rozhraní) mají navíc tu nevýhodu, že uživatel nalezená data nemůže jednoduše sdílet například se svými kolegy (od toho u webových rozhraní hodnocené reflektování stavu aplikace v adresním řádku prohlížeče). 

			Z výše uvedeného je zřejmé, že je potřeba rozhraní, které bude použitelné univerzálně na všech platformách, a zároveň bude implementovat funkcionalitu všeobecně očekávanou u rozhraní pro wordnet. Tou je jednak textová prezentace dat a k ní dodatečná vizualizace ve formě grafu. Ideální rozhraní by také mělo splňovat designové požadavky moderních webových aplikací na přístupnost, tedy například nepředpokládat určitou velikost obrazovky uživatelova zařízení či ctít zásady vizuální přístupnosti (konstrastní barevné kódování, nepřekrývání textových prvků na obrazovce etc.). Je samozřejmě evidentní, že grafická prezentace dat nemůže být přizpůsobena velikosti obrazovky mobilního zařízení, mají-li být textové prvky čitelné a zároveň zachované rozložení prvků, může však být alespoň zajištěno, že funkcionalita posouvání zobrazovací plochy dotykem (u dotykových zařízení) není znemožněna funkcionalitou svázanou s prvky na obrazovce (jak tomu často bývá u implementací grafů synsetů, jejichž uzly je možné tažením myší přesouvat po zobrazovací ploše). 

			Kýžené rozhraní by také mělo splňovat požadavky na vizuální atraktivitu, aby se uživatelé cítili komfortně při práci s takovým rozhraním. Aspekty vizuálně komfortního rozhraní jsou do značné míry subjektivní, zdá se však, že uživatelé očekávají jistou prototypičnost designového konceptu od webové stránky s určitým účelem \parencite{walker2013simple, tuch2012role}. Studie \parencite{tuch2012role} také potvrzuje, že uživatelé pozitivně vnímají webové stránky s nízkou komplexitou, což může u implementace rozhraní wordnetu být problémem, jelikož takové rozhraní inherentně musí poskytovat velké množství informací. Cílem tedy je zobrazované informace vhodně strukturovat tak, aby uživatel měl pocit, že danou stránku zná a ví, kde na ní co najde, a necítil se zahlcen množstvím informací. Detaily designu takové aplikace jsou rozebrány v kapitole \itNameRef{cha:navrh} a následujících.

	\part{Praktická část: rozhraní k sémantické síti}
	\label{part:drei}

		\chapter{Návrh rozhraní}
		\label{cha:navrh}

			% cil prakticky casti = napsat rozhrani, ktery by bylo rozumne univerzalni (da se nasadit na cokoliv, co bude davat ocekavany format dat), umoznilo by textovy i graf. zobrazeni, bylo by pouzitelny na mobilu
			% graficky zobrazeni: bylo usouzeno, ze stabilni neinteraktivni (krom zoom/panning) je lepsi, bo se da pouzit i na telefonu; zadny tooltipy, nic

			\section{Východiska}

				Cílem praktické části této práce bylo vytvořit rozhraní k sémantickým sítím typu wordnet, které by nebylo zatíženo nedostatky, jež byly pozorovány u rozhraní testovaných v předchozí části práce. Zřejmě nejrozšířenějším problémem u existujících rozhraní pro webové prohlížeče je jejich do různé míry značná problematičnost použitelnosti na mobilních zařízeních s malým displejem (u některých naprostá nepoužitelnost). Rozhraní implementovaná ve formě aplikace pro klasické počítače jsou na mobilní zařízení nepřenositelná již ze své podstaty a jejich použitelnost na těchto zařízeních nemůže být hodnocena vůbec, faktem však zůstává, že je jejich nepřenositelnost v porovnání s webovými rozhraními diskriminuje. 

				Rozhraní, které je výstupem této práce si tedy klade za cíl umožnit uživatelům přístup k datům wordnetů odkudkoliv, bez ohledu na to, zda pracují na mobilním telefonu, tabletu či klasickém počítači. Jeho návrh se snaží být technologicky nadčasový, ačkoliv v tomto aspektu budou ve světě webových technologií, jenž se neustále vyvíjí, pravděpodobně vždy existovat jisté limity. Pro rozhraní vytvořené v rámci této práce to jsou však v podstatě pouze požadavky na webový prohlížeč koncového uživatele. Uživatelův prohlížeč musí podporovat technologie ECMAScript verze 6\td{mozna jen 5, ale zatim to na nem nefunguje}\footnote{také známý jako ECMAScript 2015; všeobecně označován jako JavaScript, čehož se bude nadále držet i tato práce}, HTML verze 5 a CSS verze 3. Podrobněji se vysvětlení požadavků na verze věnuje kapitola \itNameRef{cha:techno} na straně \pageref{cha:techno}. 

				Rozhraní bylo v rámci této práce navrženo tak, aby umožňovalo zobrazení dat jak textové, tak grafické. Textové zobrazení je vhodné pro zobrazení synsetů, které obsahují velké množství potomků (hyponyma, meronyma, etc.), a mohou v grafickém rozhraní být nepřehledné. Také se textové rozhraní spíše hodí k použití na mobilních zařízeních, jelikož se lépe přizpůsobuje velikosti obrazovky. Grafické zobrazení je naopak vhodné pro rychlý přehled míry komplexity synsetu a jeho vztahů. Může sloužit především jako ilustrace daného slovního významu. Textové a grafické zobrazení se vzájemně nedoplňují, ale jsou různým zobrazením týchž dat a je na uživateli, které ze zobrazení je pro něj komfortnější či užitečnější.

			\section{Cíle nového rozhraní}

				Přesto, že existujících rozhraní je dostupných mnoho, žádné z nich nesplňuje požadavky moderní platformně nezávislé aplikace, jak bylo ukázáno v části \itNameRef{part:zwei}. Proto vznikl záměr vytvořit takové rozhraní, které by tyto požadavky (do co největší míry) splňovalo. Vzhledem k původním záměrům samotného WordNetu (a na něj navazujících projektů), mezi nimiž byly například edukativní účely, je nutné, aby rozhraní k těmto projektům bylo intuitivní, vizuálně přitažlivé a snadno přístupné. Snadná přístupnost například minimalizuje nutnost správné přípravy školních počítačů k použití takového rozhraní, což opět zvyšuje šance, že aplikace bude skutečně použita.

				S touto vizí tedy byla vytvořena webová aplikace \simplywn, která ja v dalších částech této práce prezentována.

		\chapter{Prezentace vytvořeného rozhraní}
		\label{cha:ui}

			\section{Základní představení}

				Rozhraní \simplywn uživateli při prvním otevření prezentuje prázdnou stránku s vyhledávacím formulářem v levém sloupci a nápovědou, jak začít s rozhraním pracovat, v pravém prostoru určeném k následné prezentaci dat.\footnote{Při zobrazení rozhraní na displeji užším než 768 pixelů se z levého sloupce stane horní část stránky, k pravému prostoru se pak uživatel dostane rolováním po stránce dolů. Při popisu rozhraní bude k těmto částem nadále referováno jako k levé a pravé z důvodu přehlednosti.}

				Vyhledávací formulář sestává jednak ze samotného pole pro vložení hledaného výrazu a tlačítka na spuštění vyhledávání, a jednak z rozbalovací nabídky, v níž může uživatel vybrat zdroj dat vyhledávání (typicky různojazyčné wordnety). Nad polem pro vkládání jsou také tlačítka určená k volbě reprezentace synsetu.

				Pokud je zadané slovo po spuštění vyhledávání úspěšně nalezeno, zobrazí se uživateli v levém sloupci pod vyhledávacím formulářem nabídka synsetů, v nichž se hledaný výraz vyskytoval, a v pravém prostoru podle zvolené reprezentace buďto textová data o zvoleném (případně prvním) synsetu, či jeho reprezentace grafem.

				Při přepínání vizualizace dat mezi textovou a grafickou se mění pouze obsah pravé části rozhraní, obsah levého sloupce s vyhledávacím polem a výpisem zobrazených synsetů zůstává neměnný.

				Zadá-li uživatel výraz, který se v daném wordnetu nevyskytuje, je mu po neúspěšném dokončení vyhledávání prezentována zpráva o negativním výsledku hledání. 

			\section{Textová reprezentace}

				Textový režim reprezentace dat je až na několik odchylek relativně konservativní a snaží se uživateli přinést přehledný a zároveň komplexní výpis nalezených informací. 

				Hlavní nadpis tvoří výpis slov patřících do zobrazeného synsetu, pod ním je zobrazena definice významu, cesta a další informace, jako například slovní druh či číslo základního konceptu (BCS). Cestou k synsetu je míněn seřazený seznam synsetů, přes něž je možnost se ze zobrazeného synsetu po relacích hyperonymie dostat ke kořenovému synsetu. Pod tímto blokem s informacemi uživatel nalezne výpis synsetů, s nimiž je ten zobrazený provázán sémantickými relacemi. Jednotlivé vztahy jsou rozděleny do sloupců, což přispívá k lepšímu škálovaní reprezentace sémantických relací. Uživateli v této sekci není zobrazena jedna důležitá sémantická relace, a to hyperonymie; synsety k zobrazenému synsetu nadřazené jsou totiž zobrazeny ve zmíněné cestě.

				Aby byla zvýšena interaktivita rozhraní, jsou slovní formy v hlavním nadpisu a synsety v sekci sémantických relací uživateli vypisovány jako hypertextové odkazy. Přes tyto odkazy je možné vyhledat dané slovo či přejít na příslušný synset, čímž je umožněno procházení wordnetu přes vztahy jako jsou hyponymie, meronymie, hyperonymie etc. V textovém zobrazení se vyskytují dva druhy hypertextových odkazů. První jsou v levém sloupci v seznamu synsetů nalezených při vyhledávání a slouží k volbě zobrazeného synsetu. Vzhledem k tomu, že server po vyhledávání vrací soubor všech nalezených synsetů, nevyvolává přechod po těchto odkazech nové hledání. Druhým druhem odkazů jsou zde takové, které neodkazují na data načtená po posledním vyhledávání. Vyskytují se pravé části textového rozhraní, a to v hlavním nadpisu, v cestě k zobrazenému synsetu a v sémantických relacích, do nichž tento náleží. Pokud na ně uživatel klikne, spustí se nové vyhledávání podle slovní formy nebo identifikačního řetězce synsetu, na nějž uživatel klikl. Výsledkem hledání dle identifikačního řetězce jeden konkrétní synset.

				Jednou z odchylek od tradičních postupů v textové reprezentaci dat z wordnetu je udávání příslušnosti slovní formy k významu. Tato příslušnost je tradičně reprezentována číslem za slovní formou odděleným dvojtečkou. To však může pro uživatele neznalého tradic v reprezentaci wordnetů být matoucí, a tak bylo v rámci zlepšení přístupnosti nového rozhraní \simplywn přistoupeno k zobrazení čísla významu jako horního indexu za slovní formou. To by mělo údaj učinit méně obtrusivní a intuitivně snáze pochopitelný, jelikož na horní indexy bude uživatel alespoň vizuálně spíše zvyklý než na čísla za dvojtečkou.

			\section{Grafická reprezentace}

				\begin{figure}[h]
					\centering
					\includegraphics[width=1.0\textwidth]{simplywn_graph.png}
					\caption{Grafická reprezentace dat v \simplywn}
					\label{fig:simplywn_graph}
				\end{figure}

				Grafická reprezentace nalezených dat je alternativou k textové a jejím cílem je uživateli především vizualizovat vztahy, jichž je prohlížený synset členem. Podobně jako textová reprezentace se snaží příliš se neodchylovat způsobem zobrazení od již existujících zobrazení, ale zároveň eliminovat nedostatky v použitelnosti, jimiž se většina existujících implementací vyznačuje. Jmenovitě grafické zobrazení v \simplywn vůbec nepoužívá interaktivní prvky typu informačních bublin, jež se zobrazují při přejetí myší\footnote{tyto bývají anglicky označovány jako \textit{popup tooltip}}. Tyto bubliny jsou velmi oblíbené zřejmě proto, že umožňují bez ztráty přehlednosti celého grafu uživateli přístup k podrobnějším informacím. Jejich zásadní nevýhoda ovšem tkví v tom, že na dotykových obrazovkách, jimiž bývají vybavena současná mobilní zařízení, koncept přejetí myší přes prvek neexistuje. To je z toho důvodu, že grafické rozhraní na těchto zařízení nebývá ovládáno externím technickým vybavením počítače typu myš, které pohybuje po obrazovce kursorem, ale uživatel používá dotyku svých prstů či stylu a dotek na určitém místě obrazovky znamená obvykle kliknutí. Nelze tedy tak jako u klasických počítačů rozlišit přejetí kursorem myši a kliknutí a funkcionalita zmíněných informačních bublin je tedy buďto zcela nepřístupná či přístupná jen velmi omezeně. 

				Další interaktivita značně rozšířená u existujících vizualizací je možnost přetažením uzlu myší přeuspořádat graf. U některých z vizualizací může taková možnost být užitečná proto, že při větším počtu uzlů se některé uzly překrývají a přetažením lze takový překryv eliminovat (většinou však na úkor rozložení jiné části grafu). Obecně by však při správném nastavení zobrazení taková funkcionalita neměla být potřebná, a proto nebyla zavedena ani do vizualizace v \simplywn.  
				% zminit se, ze tam nejsou ani linky, jezto to neumime zaimplementovat asi.. ale mozna jo: http://visjs.org/examples/network/events/interactionEvents.html

				Koncept zobrazení grafem spočívá v použití tří druhů uzlů s odlišným významem, které jsou spojeny hranami, jejichž význam je závislý na tom, které uzly spojují. Prvním druhem uzlů jsou ty, jež reprezentují synsety. Takový uzel se v grafu vždy vyskytuje minimálně jeden kořenový, což je v tomto případě ten synset, jehož detaily jsou otevřeny. Další jsou pak členové případných sémantických relací, jichž je kořenový členem. Z kořenového uzlu vedou hrany ke dvěma zbývajícím druhům uzlů, jimiž jsou slovní formy a sémantické relace. Uzly reprezentující slovní formy jsou se synsetem, k němuž náležejí, propojeny hranou pojmenovanou \ex{member word}, tedy členské slovo, a žádné další hrany z nich nevedou. 

				Co se uzlů reprezentujících sémantické relace týče, tam může význam jejich přítomnosti v grafu být na první pohled poněkud obskurnější, důvody k jeho použití jsou však jednoduché. Intuitivně by sémantická relace měla být hrana, jež vede mezi synsety spojenými touto relací. Pojmenování hrany by se pak odvíjelo od druhu relace, již reprezentuje. Problémem s tímto způsobem zobrazení tkví v tom, že značně centralizuje zobrazení do hvězdicovitého obrazce okolo kořenového uzlu, což zobrazované uzly vizuálně uspořádá do hierarchicky ploché struktury a sníží přehlednost grafu. Z tohoto důvodu byly do vizualizace grafem zavedeny uzly reprezentující sémantické relace, které synsety svázané s kořenovým uzlem stejným vztahem sdružují do jednoho podstromu. Díky tomu je dosaženo rozčlenění grafu na jednotlivé sekce podle těchto sémantických relací a uživatel není nucen zkoumat význam každé hrany mezi synsety (cf. vizualizace v rozhraní Cornetto, ilustrace \ref{fig:wncorplant} na straně \pageref{fig:wncorplant}).

				Responsivita vizualizace je inherentně omezena tím, že se graf byť i nepříliš komplexního synsetu nemůže na malou obrazovku mobilního zařízení vejít, zvláště mají-li být textové popisy uzlů čitelné. Tím, že vizualizace umožňuje přibližování a oddalování, je však tento problém do jisté míry eliminován, protože uživatel může graf přiblížit a přejíždět po něm podobně, jako to umožňují například aplikace na zobrazování elektronických map\footnote{\url{https://www.mapy.cz} inter alia}. Toto řešení samozřejmě zavádí na dotykových obrazovkách problém s tím, aby nebylo znemožněno rolování po zbytku stránky. To bylo vyřešeno v \simplywn zachováním okrajů po stranách kolem plochy s grafem, kde se běžně vyskytují rolovací ovládací lišty, a uživatel tak tohoto okraje může využít k rolování stránky namísto posouvání grafu.


		\chapter{Použité technologie}
		\label{cha:techno}

			Rozhraní \simplywn je vytvořeno jako aplikace pro webové prohlížeče, přesněji řečeno jde o dokument v HTML\footnote{HyperText Markup Language} (k čemuž se všeobecně referuje jako k webové stránce), jenž je dynamicky generovaný na straně klienta pomocí skriptovacího jazyka JavaScript a jím zpracovávaných dat ze serveru posílaných ve formátu JSON\footnote{JavaScript Object Notation}. Vzhled dokumentu je definován pomocí jazyka CSS\footnote{Cascading Style Sheets}. Jednotlivé role uvedených technologií budou rozebrány v následujících sekcích.

			\section{HTML}

				Značkovací jazyk HTML je základním stavebním prvkem většiny aplikací pro webové prohlížeče. Jako značkovací jazyk umožňuje HTML strukturalizaci textového dokumentu do částí, definici významů jednotlivých částí (například označení nadpisu) a do jisté míry jej lze užívat, byť se to nedoporučuje, i pro definici vzhledu textu (tučný řez písma, etc.). Poskytuje také prostředky pro tzv. hypertextové odkazy, jimiž lze provázat různé dokumenty či části jednoho dokumentu. Nezřídka se tyto odkazy v dnešní době používají jako ovládací prvky k řízení změn obsahu dynamicky generovaného dokumentu, kterým je například zde popisované rozhraní \simplywn. 

				V rozhraní \simplywn HTML slouží k definici struktury zobrazené webové stránky, jejímu rozdělení na jednotlivé sekce (levý sloupec, pravá plocha, záhlaví, etc.) a ke strukturalizaci posléze zobrazovaných informací. Příkladem může být výpis sémantických relací, v nichž je zobrazovaný synset přítomen. Jednotlivé relace jsou rozděleny do bloků (jež jsou později pomocí CSS nastaveny tak, aby se zobrazovaly jako sloupce) a jednotlivé synsety svázané se zobrazovaným příslušným vztahem jsou vypsány v seznamu. 

				\begin{figure}[h]
					\centering
					\includegraphics[width=1.0\textwidth]{html3d2.png}
					\caption{Třídimenzionální vizualizace struktury rozhraní}
					\label{fig:html3d}
				\end{figure}

				Pátá verze HTML je v aplikaci využita jakožto nejnovější s ohledem na to, že je podporovaná ve všech moderních prohlížečích. \parencite{html5support} Důležitou motivací pro použití této verze je také podpora prvku \texttt{<canvas>}, který umožňuje vykreslovat do dokumentu grafiku \parencite{w3schools2017htmlcanvas} a je pomocí něho implementována grafická reprezentace dat. 

			\section{JavaScript}

				JavaScript je vysokoúrovňový\footnote{silně abstrahovaný od nízkoúrovňových instrukcí procesoru a nezávislý (či méně závislý) na technickém vybavení počítače} dynamický\footnote{úkony, jež jsou u statických jazyků prováděny při kompilaci, jsou prováděny za běhu programu} netypovaný\footnote{kterékoliv operace jsou povoleny nad jakýmikoliv daty, nekontroluje se typ proměnných} a interpretovaný\footnote{program spouštěný bez předchozí statické kompilace, interpretem následně rozložený na kompilované podrutiny} programovací jazyk používaný vedle HTML a CSS jako jedna ze tří základních technologií pro tvorbu webových aplikací. Jeho funkce spočívá v zásadě v ovládání obsahu stránky na základě událostí, kterými mohou být například akce uživatele jako kliknutí na odkaz či akce serveru jako je dokončení požadavku a navrácení jeho výsledků.% [cit] wikipedia

				Popisované rozhraní momentálně vyžaduje JavaScript (resp. ECMAScript) ve verzi 6 a využívá jej na sledování akcí uživatele (inicializace hledání a přepínání zobrazeného synsetu či vizualizace), odesílání požadavků uživatele serveru (hledání), přijímání odpovědí na tyto požadavky a následné zpracování odpovědi a aktualizace HTML dokumentu%tady si stojim za "html dokument", bo je to "htmlovy dokument", nikoliv neco jako "jazyk html" (cf. "webovy dokument" vs. "prostredi web(u)")
				, tedy zobrazení výsledků hledání uživateli. %history api?
				Zjednodušený vývojový diagram této logiky je ukázán na ilustraci \ref{fig:wordnet-ui-diag} na straně \pageref{fig:wordnet-ui-diag}.

				\subsection{JSON}
				\label{cha:json}

					JSON je formát určený pro výměnu dat a je zároveň strojově i lidsky čitelný (jedná se o čistý text). Strukturálně se podobá objektům v JavaScriptu, tedy sestává z párů \texttt{klíč:hodnota}. Klíčem bývá řetězec znaků, hodnotou může být cokoliv, co je serializovatelné. \parencite{jsonDoc}

					Tento formát je vhodný na přenos informací, jakými jsou data z wordnetu, jelikož se jedná o strukturovaná, čistě textová data. Další možností pro přenos takových dat je například formát XML\footnote{eXtensible Markup Language}. Ten oproti formátu JSON umožňuje přesnější specifikace typů hodnot a validaci dat, avšak pro stejné množství informací je datově větší (má větší režii na značkování). XML je navíc oproti formátu JSON méně pohodlně zpracovatelné v JavaScriptu, jelikož syntax formátu JSON reflektuje strukturu objektů v JavaScriptu. Vzhledem k tomu, že data z wordnetů bývají krátké řetězce, je evidentní, že při použití XML by bylo přenášeno zbytečně velké množství strukturních informací v podobě značkování. \parencite{jsonVsXML} To v kombinaci s jednoduchostí zpracování formátu JSON v JavaScriptu přispělo k rozhodnutí použít v aplikaci \simplywn k přenosu dat JSON. Ukázka formátu JSON je v kapitole \itNameRef{cha:answerStruct} na straně \pageref{cha:answerStruct}

				\subsection{Ajax}

					Ajax\footnote{Asynchronous JavaScript And XML} je souhrnné označení skupiny technologií používaných pro asynchronní přenos dat mezi webovým prohlížečem a serverem, tedy pro komunikaci webových aplikací se serverem. Rozdíl oproti tradičnímu přenosu informací mezi serverem a webovou stránkou načtenou u klienta tkví v tom, že bez použití Ajaxu je nutné pro zpracování odpovědi na požadavek načíst stránku znova. Hlavním využitím Ajaxu bývají právě aplikace, jejichž funkčnost je závislá na komunikaci se vzdáleným serverem bez nutnosti načítat celou webovou stránku znovu. \parencite{garrett2005ajax}

					U komplexních webových aplikací může být načítání celé stránky znova nevýhodné z mnoha důvodů. Jedním ze scénářů může být například potřeba obnovit data na části stránky, zatímco uživatel vkládá data (například komentář) na jiné části\footnote{Existuje jistě možnost toto obejít pomocí rámců, ale ty mají mnoho zásadních nevýhod.}. Úplným znovunačtením by byla uživatelova vložená data ztracena, takže je nutné obnovovat jen její část, k čemuž je výhodné využít Ajax.

					Dalším důvodem k využívání Ajaxu je uživatelův komfort -- je možné během načítání zobrazovat vlastní indikátor načítání, při výpadku spojení zobrazit uživatelsky přívětivou zprávu, etc.

					Asynchronní výměna dat mezi webovým prohlížečem a serverem serverem probíhá pomocí aplikačního rozhraní XMLHttpRequest (XHR) dostupného prohlížečům pomocí JavaScriptu. \parencite{mozilla2017XMLHTTP}

				\subsection{Historie v prohlížeči}

					Historie je schopnost webových prohlížečů přecházet mezi jednotlivými stavy, které nastaly během uživatelova užívání daného okna (či karty) prohlížeče. Klasicky jsou těmito stavy jednotlivé stránky, které uživatel navštívil, ať už v rámci jednoho nebo různých portálů. Pokud například navštíví domovskou stránku vyhledávače DuckDuckGo\footnote{\url{https://duckduckgo.com}}, vyhledá výraz \ex{brno} a následně otevře například výsledek hledání vedoucí na portál Wikipedia\footnote{\url{https://en.wikipedia.org/wiki/Brno}}, zaznamená prohlížeč automaticky tři stavy (domovská stránka vyhledávače, stránka s výsledky hledání, stránka na Wikipedii), k nimž je možné se vrátit. Pro porovnání s nastíněním problému, jež následuje níže, je dlužno podotknout, že v obou zmíněných přechodech mezi stavy karty prohlížeče se stránka načítá zcela znovu\footnote{stav k 1. 6. 2017}, takže se změna stavu zaznamenává automaticky (netřeba ji ošetřovat na straně aplikace).

					Při užívání dynamického generování obsahu stránky na straně klienta a přenosu požadavků a dat pomocí Ajaxu nastává problém, že z hlediska prohlížeče se stav nezmění, tudíž není nic zapsáno do historie. Tento problém byl v minulosti řešen různými způsoby, pátá verze HTML však implementuje aplikační rozhraní pro možnost ovládat uměle historii prohlížeče. Je tedy možné v libovolný okamžik běhu JavaScriptu přidat do historie další stav a tím umožnit jeho obnovení například při uživatelově akci \uv{přejít zpět}.

					Druhým problémem při dynamickém generování obsahu stránky je udržování adresního řádku prohlížeče, který obsahuje URL\footnote{Uniform Resource Locator}, aby bylo později možné z adresy obnovit stav; to je výhodné zejména z toho důvodu, že by mělo být možné adresu uložit či sdílet. JavaScript umožňuje nativně manipulaci s adresou v adresním řádku, přičemž \simplywn její udržování implementuje pomocí knihovny URI.js. \parencite{urijsWeb}

				\subsection{Použité knihovny a zásuvné moduly pro JavaScript}

					\paragraph{jQuery} je knihovna navržená s cílem usnadnit programování v JavaScriptu. Je zaměřená především na zjednodušení práce s elementy v HTML dokumentu, práci s~událostmi (např. stisk klávesy) a~komunikaci mezi klientem a~serverem pomocí Ajaxu. \parencite{jqueryWeb}

					\paragraph{URI.js} zjednodušuje práci s adresami (URL). Knihovna URI.js umožňuje jednak obsah adresního řádku zpracovávat pro další práci, a jednak jej umožňuje nastavovat tak, aby odpovídal stavu, v němž se aplikace nachází. \parencite{urijsWeb}

					\paragraph{Perfect Scrollbar} je zásuvný modul určený k zobrazování rolovacích lišt, jež jsou flexibilnější než základní systémové lišty a nezávislé na operačním systému. Dají se na rozdíl od těch systémových mimo jiné vzhledově přizpůsobit zbytku aplikace, a tudíž nepůsobí tak obtrusivně. \parencite{perfectScrollbarGithub}

					\paragraph{Vis.js} je vizualizační knihovna, navržená pro zobrazení velkého množství dynamicky generovaných dat. Umožňuje navíc pokročilou manipulaci s daty, jako je přidávání a odebírání uzlů grafu, změny štítků uzlů, etc. V \simplywn je využit její modul \textit{Network} k zobrazení detailů synsetu grafem. \parencite{visjWeb}

			\section{CSS}

				CSS, neboli kaskádové styly, je jazyk pro definování vzhledu prvků (obvykle náležejících do HTML dokumentu). Pojem kaskádový má tento jazyk v názvu proto, že vlastnosti prvků jsou děděny od rodičů a podle dědičnosti je prvky možné i vybírat. Prvotním smyslem užívání CSS bylo oddělení obsahu webové stránky od jejího vzhledu. To je podstatné pro udržitelnou správu stránky, větší flexibilitu a sémantiku zdrojového kódu. Použití CSS umožňuje zobrazit jednu stránku různými způsoby, jedním z~nichž může být například vzhled přizpůsobený pro tisk či pro barvoslepé uživatele.

				Kaskádové styly jsou v \simplywn tedy využity k tomu, aby byla stránka s aplikací uživatelsky přívětivá, atraktivní a vhodně zobrazena na zařízeních různých velikostí (částečně je využito stylování podmíněného šířkou obrazovky). Třetí verze CSS je využita proto, že je nejnovější, je vyžadována pro kvalitní implementaci responsivity rozhraní a také ji jako takovou využívá Bootstrap. 

				\subsection{Bootstrap}

					Aplikace \simplywn využívá Bootstrap\footnote{\url{http://getbootstrap.com/}}, což je sada nástrojů v HTML a CSS pro tvorbu webových stránek. Obsahuje šablony různých prvků, které se na webových stránkách vyskytují, definuje základní vlastnosti některých prvků, aby rozložení stránky bylo responsivní a mimo jiné také do jisté míry zajišťuje konformní vzhled napříč různými prohlížeči.

		\chapter{Implementace rozhraní}

			Tato kapitola se bude věnovat funkcionalitě rozhraní a jejímu technickému provedení. Rozebere, jak aplikace získává data od uživatele o tom, co má být vyhledáno, jak vypadá odpověď serveru s výsledky, již dostává zpět, a jak tuto odpověď zpracovává. Jejím cílem je přiblížit čtenáři způsob, jak celá aplikace pracuje.


			\begin{figure}[h]
				\centering
				\includegraphics[width=1.0\textwidth]{wordnet-ui-diag_v2.eps}
				\caption{Zjednodušený vývojový diagram logiky rozhraní}
				\label{fig:wordnet-ui-diag}
			\end{figure}

			\section{Získávání dat}
				
				Data, jež jsou ve výsledku zobrazována uživateli, jsou načítána ze serveru v závislosti na vstupu, který uživatel tím či oním způsobem zadal. Vstup uživatele může být dvojího druhu (byť z hlediska výsledného zpracování se nijak neliší). Prvním je, že uživatel zadává kýžené slovo do vyhledávacího pole, druhým, že klikne na odkaz vedoucí na jiný synset. Druhý případ se potom od prvního liší pouze v tom, že text ve vyhledávacím poli je buď identifikátor, nebo členské slovo synsetu\footnote{podrobněji o tomto rozdílu v [KDE O TOM KURVA PISU?! nemuzu to najit]}, na nějž bylo kliknuto. Jde tedy o jistou zkratku, během níž se automaticky zadá identifikátor daného synsetu do vyhledávacího pole a spustí se vyhledávání.

				\subsection{Parametry vyhledávání}
				
					Aby bylo možno vyhledávání spustit se správnými parametry, je nutno získat od uživatele dva údaje. Jedním je vyhledávané slovo, které je extrahováno z vyhledávacího pole, druhým je zdroj dat, v němž se má vyhledávat. Pro výběr druhého slouží rozbalovací nabídka pod vyhledávacím polem, která obsahuje seznam zdrojů dat, v nichž je možno vyhledávat.

					Zde je na místě podotknout, že schopnost aplikačního rozhraní, jež zajišťuje samotné vyhledávání, vyhledávat jak podle slovní formy, tak dle identifikátoru, otevírá možnost vyhledávat konkrétní synset napříč dostupnými wordnety. 

					Po spuštění vyhledávání je na server s aplikačním rozhraním odeslán požadavek složený ze vstupních dat. Jeho forma je následující:

					\medskip
					\url{https://nlp.fi.muni.cz/~xrambous/fw/abulafia/ZDROJ?action=jsonvis&query=SLOVO}\hspace{1em},
					\medskip

					kde \texttt{ZDROJ} zastupuje prohledávaný wordnet a \texttt{SLOVO} zastupuje vyhledávaný výraz.

				\subsection{Struktura odpovědi serveru}
				\label{cha:answerStruct}

					V případě úspěšného hledání je ze serveru přijata odpověď ve formátu JSON (bližší popis formátu v kapitole \itNameRef{cha:json} na straně \pageref{cha:json}). Jeho strukturu lze popsat jako pole\footnote{Seřazená množina prvků, v terminologii jiných jazyků například seznam (\textit{list})} objektů\footnote{Objekt je množina dvojic \texttt{klíč:hodnota}, v terminologii jiných jazyků je tento datový typ nazýván slovníkem (\textit{dictionary}), asociativním polem (\textit{associative array}), etc.}, kde každý z těchto objektů reprezentuje jeden nalezený synset. Objekty synsetů pak obsahují jednak řetězce jako identifikátor, slovní druh či definici, a jednak další pole. Pole pod klíčem \texttt{paths} slouží k ukládání cesty k danému synsetu, pole pod klíčem \texttt{synset} obsahuje jeho členská slova a pole s klíčem \texttt{children}\footnote{Klíč \texttt{children} se užívá všeobecně k označení pole s potomky.} pak obsahuje sémantické relace daného synsetu. 

					Struktura odpovědi může být ilustrována následujícím zjednodušenou ukázkou:

					\begin{verbnobox}[\verbarg\small]
[
    { // hlavní synset:
        "id": "ENG20-02759263-n",
        "pos": "n",
        "def": "definice synsetu",
        "synset": [
            {
                "name": "kotouč",
                "meaning": "2"
            },
            ... // další členská slova synsetu
        ], // cesty k synsetu:
        "paths": [
            {
                "name": "path-1",
                "breadcrumbs": [
                    {
                        "id": "ENG20-00020486-n",
                        "def": "",
                        "synset": [
                            {
                                "name": "abstrakce",
                                "meaning": "1"
                            },
                            ... // další členská slova synsetu
                        ]
                    },
                    ... // další části cesty
                ]
            },
            ... // alternativní cesty
        ],
        "children": [
            { // sémantické relace synsetu
                "name": "hyponym",
                "children": [
                    {
                        "id": "ENG20-04050919-n",
                        "def": "",
                        "synset": [
                            {
                                "name": "člunek",
                                "meaning": "1"
                            },
                            ... // další členská slova synsetu
                        ]
                    },
                    ... // další synsety v dané relaci
                ]
            },
            ... // další sémantické relace
        ]
    },
    ... // další nalezené synsety
]
					\end{verbnobox}

					Pokud je vyhledávání neúspěšné, vrací server pouze prázdné pole. Tento stav je kontrolován aplikací a v případě, že nastane, je uživatel upozorněn, že dané hledání nevrátilo pro zvolený zdroj dat žádné výsledky. 

			\section{Zpracování odpovědi}
			\label{cha:zpracovani}

				Získaná data jsou v několika krocích zpracována tak, aby jejich použití v aplikaci bylo jednodušší a jejich struktura odpovídala potřebám rozhraní. Jmenovitě je pole objektů se synsety serializováno do jednoho objektu, v němž lze odkazovat na potřebný synset univerzálnějším způsobem (podle jeho identifikátoru). 

				Textová reprezentace nevyžaduje další reorganizaci dat, ale zpracování odpovědi pro potřeby vizualizace grafem je poněkud komplexnější. Je evidentní, že pro vykreslení grafu je nutné mít k disposici uzly a hrany, jimiž jsou uzly spojeny. V JSONu získaném ze serveru jsou tyto vztahy obsaženy pouze implicitně tak, že určitý objekt je rodičem dalších objektů, či dítětem výše postaveného objektu. Explicitní údaje o těchto vztazích ve vlastnostech objektů nejsou, a je tudíž nutné je pro knihovnu, která graf vykresluje, vytvořit. Pro tento účel obsahuje aplikace \simplywn funkci \texttt{DFSThruSynsets()}, která prochází synsety v JSONu metodou procházení do hloubky%\footnote{Procházení grafu do šířky je algoritmus pro prohledávání grafu, jenž prochází nejprve sousední uzly a až poté, co projde všechny, se posouvá na další úroveň grafu \parencite{epstein1996ics}}
				\footnote{Procházení do hloubky (DFS, depth first search) je algoritmus pro prohledávání grafu, jenž prochází postupně celé větve grafu.  Druhou možností je procházení do šířky (BFS, breadth first search), které prochází nejprve všechny uzly co nejblíže kořeni (tedy na stejné úrovni) a až poté přechází na nižší (hlubší) úroveň. \parencite{epstein1996ics} Z hlediska aplikace \simplywn se tyto dva přístupy liší jen tím, že vykreslují mírně odlišně uspořádané grafy (struktura je zachována)}, každý nalezený synset zapisuje do seznamu uzlů a zároveň k němu do seznamu hran zapisuje uspořádanou dvojici, která sestává z identifikátoru jeho rodičovského uzlu a jeho vlastního identifikátoru. Pořadí, v jakém jsou jednotlivé uzly procházeny, je znázorněno na grafu \ref{fig:bfs-ilustr} na straně \pageref{fig:bfs-ilustr}. Procházení do hloubky je implementováno pomocí zásobníku a rekurse. Během procházení dat je pomocí různých atributů nalezených objektů rozlišováno pět druhů uzlů, a to kořenový synset (\texttt{root}), tedy ten, jejž si uživatel vybral, jeho členská slova (\texttt{rootLeaf}), ostatní synsety (\texttt{synset}), jejich členská slova (\texttt{leaf}) a uzly představující sémantické relace (\texttt{semGroup}). 



				% \begin{tikzpicture}[sibling distance=10em,
				%   every node/.style = {shape=rectangle, rounded corners,
				%     draw, align=center,
				%     top color=white, bottom color=blue!20}]]
				\begin{figure}
					\centering
					\begin{tikzpicture}[level distance=2em]
					  \tikzstyle{every node}=[shape=rectangle, rounded corners, draw, align=center]
					  \tikzstyle{level 1}=[sibling distance=10em]
					  \tikzstyle{level 2}=[sibling distance=5em]
					  \tikzstyle{level 3}=[sibling distance=3em] 
					  \node {1}
					  	child { node{8} 
					  		child { node{10} }
					  		child { node{9} }
					  	}
					  	child { node{4} 
					  		child { node{7} }
					  		child { node{6} }
					  		child { node{5} }
					  	}
					  	child { node{2} 
					  		child { node{3} }
					  	};
					\end{tikzpicture}
					\caption{Ilustrace pořadí, v jakém jsou uzly zapisovány při procházení grafu do šířky}
					\label{fig:bfs-ilustr}
				\end{figure}

			\section{Zobrazení dat}

				Ve chvíli, kdy jsou data zpracována, může být jejich příslušná část, tedy jeden synset, zobrazena uživateli. Který synset se bude zobrazovat, je v případě, že v adrese (URL) existuje parametr synset specifikující, rozhodnuto na základě onoho parametru, jinak je vybrán pro další operace objekt se synsetem, jenž se v datech vyskytuje jako první. 

				Zároveň je uživateli v levém sloupci zobrazen seznam nalezených synsetů, tedy výpis jejich členských slov. Každý řádek je zároveň odkazem, kliknutím na nějž si uživatel může vybrat, kterou část dat si přeje prohlížet. Taková akce vyvolá výběr objektu, který obsahuje synset s identifikátorem shodným s tím, který byl uveden v odkazu.

				Objekt obsahující příslušnou část dat je následně předán funkci \texttt{renderView()}, která na základě stavu aplikace uloženého v adrese zavolá funkci zobrazující data textovou reprezentací, či grafickou.

				\paragraph{Textové zobrazení} dat je řešeno jednoduchým naplňováním částečně předpřipraveného HTML dokumentu očekávanými daty. Rozhraní počítá s neúplností dat, takže není v tomto ohledu striktní a pokud se požadované informace v datech nevyskytují (typicky definice je hodnota, která u nezanedbatelného počtu konceptů v českém wordnetu chybí), je daná struktura naplněna prázdným řetězcem a nezobrazuje se. Kromě informací, jejichž existence je v datech nejistá, se v nich vyskytují i informace s předem neznámou délkou. Mezi takové patří například počet synsetů odpovídajících hledanému výrazu, pro každý z nich pak počet relací, jichž jsou členem, počet dalších synsetů v dané relaci či počet členských slov jednotlivých zobrazených synsetů. Zobrazení takových informací je v rozhraní řešeno cykly a z hlediska struktury HTML to bývají seznamy, jakožto prvky nejvhodnější pro výpis položek. Výjimkou jsou členská slova synsetů, ta jsou vypisována jako řetězce znaků rozdělené čárkami (pomocí funkce \texttt{synString()}).

				\paragraph{Grafické zobrazení} realizuje knihovna Vis.js, které jsou vstupem data vytvořená ze serverové odpovědi (struktura a tvorba těchto dat je popsána v kapitole \itNameRef{cha:zpracovani} na straně \pageref{cha:zpracovani}). Nastavení vykreslovací funkce je upraveno tak, aby výsledný graf byl statický a s jeho uzly nebylo možné tažením myší hýbat. Důvody k tomuto nastavení jsou rozebrány v kapitole \itNameRef{cha:navrh} na straně \pageref{cha:shrnuti-prehledu}. Mezi dalšími nastaveními jsou například parametry zajišťující penalizaci překryvu uzlů (nutno podotknout, že jakkoliv vysoká tato penalizace může být, nepodařilo se najít taková nastavení, aby k ní za žádných okolností -- zejména na velkých synsetech -- s jistotou nedošlo), vzhled jednotlivých druhů uzlů\footnote{vzhled prvků grafu není možné definovat externími pravidly v CSS} či parametry stabilizace grafu. Stabilizace grafu je proces, který probíhá poté, co jsou knihovnou vytvořeny na vykreslovacím plátně všechny uzly a hrany mezi nimi. Vykreslovací funkce využívá emulace fyzikálních zákonů mezi jednotlivými prvky, takže uzly se odpuzují podobně, jako v reálném světě se odpuzují elektrony, a hrany se chovají jako pružiny, které mají svou ideální délku. Kombinace odpudivých a přitažlivých sil zajišťuje tendenci k rovnoměrnému rozmístění prvků na ploše.

			\section{Uchovávání stavu v URL}

				Aplikace svůj stav reflektuje v URL a umožňuje tak případné pozdější obnovení do identického stavu, v jakém se v libovolném okamžiku právě nachází. Aby bylo toto dosažitelné, je nutné uchovávat zvolený zdroj dat, obsah hledacího pole, zvolenou část výsledků pro zobrazení (levý sloupec) a aktivní režim zobrazení dat. 

				Vzhledem k tomu, že se ve všech případech jedná o relativně krátké textové informace, lze je pohodlně ukládat jako parametry URL\footnote{běžně označované jako query strings}. Ty jsou nepovinnou částí adresy a mohou obsahovat data ve formátu \texttt{klíč:hodnota}, například \url{http://www.example.org/index.html?klic=hodnota}. \parencite{berners2005uniform} Ačkoliv \textcite{berners2005uniform} nespecifikuje použití více dvojic \texttt{klíč:hodnota} v parametrech URL, běžně se toto implementuje oddělováním jednotlivých dvojic znakem ampersand (\texttt{\&}). Tak je řešena problematika ukládání více údajů i rozhraní \simplywn: 

				\medskip
				\url{...wordnet-ui.html?input=kolo&source=wncz&vis=graph#ENG20-03403613-n}
				\medskip

				Další částí adresy obsahující informace o stavu aplikace je fragment, tedy řetězec znaků za znakem křížku (\texttt{\#}). Fragment je podobně jako parametry nepovinná část adresy, která se využívá na bližší identifikaci informace v dokumentu, tedy může být kupříkladu odkazem na podsekci v dokumentu. \parencite{berners2005uniform} Všeobecně se fragment používá k odkazování uvnitř dokumentu lokálně na straně uživatele, naproti tomu parametry bývají odesílány na server pro účely parametrizace požadavků. V rozhraní \simplywn je fragment využíván k ukládání identifikátoru zvoleného synsetu, jeho změna tedy nevyvolává proces hledání a komunikace se serverem.

				Při načtení v prohlížeči aplikace nejprve zjišťuje, zda adresa obsahuje parametry, ze kterých by se měl obnovit stav. Pokud ne (tedy neobsahuje žádné parametry, či se existující parametry neshodují s těmi používanými aplikací), je rozhraní uvedeno do výchozího stavu. V opačném případě proběhne pokus o uvedení rozhraní do stavu odpovídajícího hodnotám parametrů, tedy spustí se vyhledávání podle parametru \texttt{input} ve wordnetu zvoleném podle parametru \texttt{source}. Pokud hledání proběhne úspěšně, zobrazí se v režimu vizualizace podle parametru \texttt{vis} synset, jehož identifikátor odpovídá fragmentu.

		\chapter{Možnosti dalšího vývoje}
		\label{cha:co-se-nestihlo}

			Ačkoliv je v rámci této práce vytvořená implementace funkčním rozhraním k datům wordnetů, skýtá zároveň široké možnosti ohledně vylepšení, kterými by mohla být obohacena. Tato kapitola stručně probere ty, které byly identifikovány jako nejzajímavější pro zvýšení reálné použitelnosti rozhraní \simplywn. 

			Jelikož už nyní aplikace umožňuje pracovat s více wordnety, je evidentní, že by bylo vhodné dát uživateli možnost přepnout jazyk rozhraní do jeho vlastního jazyka. Množství textů, které by bylo potřeba lokalizovat, sice není nijak enormní, navíc dosavadní jazyk rozhraní je angličtina jakožto lingua franca, přesto by však lokalizace i těch několika řetězců, jež se v rozhraní vyskytují, mohla mít potenciál zvýšit komfort uživatele při používání rozhraní.

			S tímto souvisí i další krok, který by jistě vedl ke zvýšení komfortu laického koncového uživatele. V rozhraní se momentálně vyskytují názvy sémantických relací přímo ve formě ze serverové odpovědi, která vychází z lingvistických názvů příslušných vztahů. Bylo by tedy vhodné zajistit jejich překlad do přirozeného jazyka tak, aby i uživatel bez lingvistického vzdělání byl schopen identifikovat, o jakou relaci jde. Lze například předpokládat, že nežli pojem \ex{hyponymum} bude takovému uživateli srozumitelnější označení \ex{Významy podřadné}, etc.

			Samotné vyhledávání by bylo možné usnadnit uživateli tak, že by mu byly na základě částečného vstupu nabízeny výrazy, jež se ve zvoleném wordnetu vyskytují. Takovou nápovědu používá například rozhraní WordVis. \parencite{wordvis2010vercruysse}

			Pokud by rozhraní mělo být využito jako výukového materiálu, nebylo by od věci uvažovat o jeho propojení například s Internetovou jazykovou příručkou\footnote{\url{http://prirucka.ujc.cas.cz}}, z níž by bylo možné čerpat definice či slovní formy.

			Dalším směrem, kterým by se budoucí vývoj mohl ubírat, je rozšíření podpory více jazyků. V současné implementaci sice již existuje možnost přepínat jazyky hledání (pomocí přepínání zdrojů dat) a je i možné díky identifikátorům synsetů jednotným přes více jazyků vyhledávat jeden koncept v různých jazycích, bylo by však vhodné umožnit jednodušší a rychlejší cestu, jak procházet významy slov v mezijazykovém kontextu.

			V neposlední řadě jsou možnosti rozvoje i v detailech týkajících se uživatelské zkušenosti s rozhraním. V tomto ohledu se naskýtají možnosti rozšíření schopností aplikace obsloužit uživatele různými způsoby znevýhodněnými, ať už jde o jejich fyzické znevýhodnění (například poškození zraku) či znevýhodnění v oblasti programového vybavení jejich zařízení (tedy omezená podpora technologií vyžadovaných pro správný běh aplikace, zejména JavaScriptu). Nutno podotknout, že co se prvého týče, byla během návrhu grafické stránky rozhraní snaha o co nejvyšší přístupnost, takže byl zvolen vzhled s vysokým kontrastem kontur a nezaváděly se ani žádné prvky závislé na pohybu kursoru myši po obrazovce. 

		\chapter*{Závěr}\label{zaver}

		\addcontentsline{toc}{chapter}{Závěr}

			Cílem praktické části této práce bylo vytvořit funkční uživatelské rozhraní k sémantickým sítím typu wordnet. Vlastnosti, které by mělo takové rozhraní mít, byly určeny na základě podrobného empirického rozboru některých existujících rozhraní a rozebrány v kapitole \itNameRef{cha:navrh}. Byly určeny klíčové nedostatky testovaných rozhraní vzhledem k rozvoji technologií a trendům v oblasti způsobu využívání rozhraní zpřístupňujících data. Jako nejrozšířenější nedostatek existujících rozhraní byla identifikována jejich nedostatečná univerzalita, tedy schopnost obsloužit uživatele bez ohledu na to, zda právě pro přístup k danému rozhraní chtějí použít mobilní zařízení s relativně malou obrazovkou, tablet či klasický počítač a bez ohledu na to, který operační systém na svém zařízení používají. Přitom nezávislost na zařízení a platformě je z dnešního hlediska klíčovou vlastností, ježto poměr uživatelů, kteří pracují na mobilním zařízení a těch, kteří pracují na klasickém počítači, se stále posouvá ve prospěch těch prvních.

			Na základě zhodnocení potřebných parametrů rozhraní, které by nebylo zatíženo vyvstavšími nedostatky testovaných rozhraní, bylo definováno, že nové rozhraní by mělo být tzv. webové, tedy zobrazitelné ve standardních prohlížečích určených k procházení sítě WWW. To jednak eliminuje potřebu uživatele přesvědčit k instalaci přídavného programového vybavení na jeho počítač, a jednak je tím zajištěna nezávislost na platformě (operačním systému), kterou uživatel používá. 

			Další podstatnou vlastností související s univerzalitou, které by nové rozhraní mělo mít, aby přineslo oproti existujícím rozhraním přidanou hodnotu, je schopnost pracovat s více zdroji informací (wordnety) a mít otevřený kód. Schopnost práce s více zdroji informací implikuje univerzální formát vstupních dat, což je v kombinaci s otevřeným kódem důležité proto, aby bylo možné rozhraní nasadit nezávisle na již existujících instalacích, modifikovat podle konkrétních potřeb a ideálně též vylepšovat. Všechny komponenty rozhraní, kterých je pro jeho funkci využíváno (jQuery, vis.js, etc.) používají licenci MIT, což umožňuje rozhraní vytvořené v rámci této práce bez omezení redistribuovat a modifikovat. 

			Po analýze všech těchto požadavků bylo vytvořeno rozhraní určené pro webové prohlížeče, které na zařízení uživatele neklade žádné nestandardní požadavky ani po stránce technického, ani programového vybavení. Aby byla splněna podmínka vyžadující možnost užívat uživatelské rozhraní aplikace na zařízení (téměř) jakékoliv velikosti, bylo toto vyvinuto za použití moderních technologií a koncipováno jako tzv. responsivní, tedy schopné se dynamicky přizpůsobit velikosti zobrazovací plochy beze ztráty čitelnosti obsahu a jednoduchosti navigace v něm. Vzhled rozhraní je minimalistický, aby se uživatel mohl zaměřit na obsah. Zároveň se však snaží být elegantní a nadčasový.

			Jako každý projekt, který se zabývá programovým vývojem, ani tento není bez možností dalšího pokračování. Možné směry v tomto ohledu naznačené v kapitole \itNameRef{cha:co-se-nestihlo} bezpochyby nejsou jediné, kterými by se rozvoj aplikace mohl ubírat. Vzhledem k tomu, že již nyní aplikace umožňuje práci s různojazyčnými sémantickými sítěmi, se zdá nejbližším dalším vhodným cílem být implementace lokalizace rozhraní do dalších jazyků a překlad lingvistických termínů do pojmů laickými uživateli srozumitelných.

			Jak bylo naznačeno v teoretické části této práce, wordnety navzdory svému původnímu určení mohou teoreticky sloužit lidským uživatelům. K tomu je ovšem potřebné rozhraní, které bude použitelné nejen pro lingvisticky vzdělané uživatele a bude odpovídat současným trendům technologickým i vizuálním. Praktická část této práce dokázala navrhnout a realizovat takové rozhraní, jež je vhodným základem pro další kroky směrem ke zpřístupnění wordnetů lidským uživatelům, zejména z řad studentů.



		% - ziskani dat od uzivatele z formulare
		% - vytvoreni url pro pozadavek na server
		% - ziskani dat ze serveru
		% - serializace dat, popis struktury jsonu
		% - vypsani seznamu nalezenych synsetu, highlight prislusneho z url / prvniho
		% - zobrazeni prislusnou vizualizaci data (podle url/prvni synset)
		% - text:
		% 	- naplneni html prislusnymi hodnotami (dynamicke generovani prvku -- tohle se jeste bude menit asi)
		% - vis:
		% 	- WNTree(), prochazeni jsonu do sirky, generovani uzlu a hran
		% 	- strucne nastaveni knihovny

		% - ilustrace html struktury?
		% - podminkovani procesu na zaklade url
		% - event listenery na odkazy



		% popis, jak to funguje
		% - struktura stranky v html?
		% - nejakej runtime JS (tezky, neni to tak pekne primocary jako baka)
		% - popis struktury JSON odpovedi

		% uvahy, co dal
		% - zrychlit?
		% - klikatelny sunsety ve stromecku?
		% - history by mohla fungovat, snad se ji podari zprovoznit
			


 
	% \begin{spacing}{1.05}
	\printbibliography[]
	% \printbibliography[title={Seznam literatury}]
	% \bibliographystyle{alpha}
	% \bibliography{bibliografie}
	% \end{spacing}

	\addcontentsline{toc}{chapter}{Bibliografie}

\end{document}